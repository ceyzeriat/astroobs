% Generated by Sphinx.
\def\sphinxdocclass{report}
\documentclass[letterpaper,10pt,english]{sphinxmanual}
\usepackage[utf8]{inputenc}
\DeclareUnicodeCharacter{00A0}{\nobreakspace}
\usepackage{cmap}
\usepackage[T1]{fontenc}
\usepackage{babel}
\usepackage{times}
\usepackage[Sonny]{fncychap}
\usepackage{longtable}
\usepackage{sphinx}
\usepackage{multirow}

\addto\captionsenglish{\renewcommand{\figurename}{Fig. }}
\addto\captionsenglish{\renewcommand{\tablename}{Table }}
\floatname{literal-block}{Listing }



\title{astroobs Documentation}
\date{August 05, 2016}
\release{}
\author{Author}
\newcommand{\sphinxlogo}{}
\renewcommand{\releasename}{Release}
\makeindex

\makeatletter
\def\PYG@reset{\let\PYG@it=\relax \let\PYG@bf=\relax%
    \let\PYG@ul=\relax \let\PYG@tc=\relax%
    \let\PYG@bc=\relax \let\PYG@ff=\relax}
\def\PYG@tok#1{\csname PYG@tok@#1\endcsname}
\def\PYG@toks#1+{\ifx\relax#1\empty\else%
    \PYG@tok{#1}\expandafter\PYG@toks\fi}
\def\PYG@do#1{\PYG@bc{\PYG@tc{\PYG@ul{%
    \PYG@it{\PYG@bf{\PYG@ff{#1}}}}}}}
\def\PYG#1#2{\PYG@reset\PYG@toks#1+\relax+\PYG@do{#2}}

\expandafter\def\csname PYG@tok@gd\endcsname{\def\PYG@tc##1{\textcolor[rgb]{0.63,0.00,0.00}{##1}}}
\expandafter\def\csname PYG@tok@gu\endcsname{\let\PYG@bf=\textbf\def\PYG@tc##1{\textcolor[rgb]{0.50,0.00,0.50}{##1}}}
\expandafter\def\csname PYG@tok@gt\endcsname{\def\PYG@tc##1{\textcolor[rgb]{0.00,0.27,0.87}{##1}}}
\expandafter\def\csname PYG@tok@gs\endcsname{\let\PYG@bf=\textbf}
\expandafter\def\csname PYG@tok@gr\endcsname{\def\PYG@tc##1{\textcolor[rgb]{1.00,0.00,0.00}{##1}}}
\expandafter\def\csname PYG@tok@cm\endcsname{\let\PYG@it=\textit\def\PYG@tc##1{\textcolor[rgb]{0.25,0.50,0.56}{##1}}}
\expandafter\def\csname PYG@tok@vg\endcsname{\def\PYG@tc##1{\textcolor[rgb]{0.73,0.38,0.84}{##1}}}
\expandafter\def\csname PYG@tok@m\endcsname{\def\PYG@tc##1{\textcolor[rgb]{0.13,0.50,0.31}{##1}}}
\expandafter\def\csname PYG@tok@mh\endcsname{\def\PYG@tc##1{\textcolor[rgb]{0.13,0.50,0.31}{##1}}}
\expandafter\def\csname PYG@tok@cs\endcsname{\def\PYG@tc##1{\textcolor[rgb]{0.25,0.50,0.56}{##1}}\def\PYG@bc##1{\setlength{\fboxsep}{0pt}\colorbox[rgb]{1.00,0.94,0.94}{\strut ##1}}}
\expandafter\def\csname PYG@tok@ge\endcsname{\let\PYG@it=\textit}
\expandafter\def\csname PYG@tok@vc\endcsname{\def\PYG@tc##1{\textcolor[rgb]{0.73,0.38,0.84}{##1}}}
\expandafter\def\csname PYG@tok@il\endcsname{\def\PYG@tc##1{\textcolor[rgb]{0.13,0.50,0.31}{##1}}}
\expandafter\def\csname PYG@tok@go\endcsname{\def\PYG@tc##1{\textcolor[rgb]{0.20,0.20,0.20}{##1}}}
\expandafter\def\csname PYG@tok@cp\endcsname{\def\PYG@tc##1{\textcolor[rgb]{0.00,0.44,0.13}{##1}}}
\expandafter\def\csname PYG@tok@gi\endcsname{\def\PYG@tc##1{\textcolor[rgb]{0.00,0.63,0.00}{##1}}}
\expandafter\def\csname PYG@tok@gh\endcsname{\let\PYG@bf=\textbf\def\PYG@tc##1{\textcolor[rgb]{0.00,0.00,0.50}{##1}}}
\expandafter\def\csname PYG@tok@ni\endcsname{\let\PYG@bf=\textbf\def\PYG@tc##1{\textcolor[rgb]{0.84,0.33,0.22}{##1}}}
\expandafter\def\csname PYG@tok@nl\endcsname{\let\PYG@bf=\textbf\def\PYG@tc##1{\textcolor[rgb]{0.00,0.13,0.44}{##1}}}
\expandafter\def\csname PYG@tok@nn\endcsname{\let\PYG@bf=\textbf\def\PYG@tc##1{\textcolor[rgb]{0.05,0.52,0.71}{##1}}}
\expandafter\def\csname PYG@tok@no\endcsname{\def\PYG@tc##1{\textcolor[rgb]{0.38,0.68,0.84}{##1}}}
\expandafter\def\csname PYG@tok@na\endcsname{\def\PYG@tc##1{\textcolor[rgb]{0.25,0.44,0.63}{##1}}}
\expandafter\def\csname PYG@tok@nb\endcsname{\def\PYG@tc##1{\textcolor[rgb]{0.00,0.44,0.13}{##1}}}
\expandafter\def\csname PYG@tok@nc\endcsname{\let\PYG@bf=\textbf\def\PYG@tc##1{\textcolor[rgb]{0.05,0.52,0.71}{##1}}}
\expandafter\def\csname PYG@tok@nd\endcsname{\let\PYG@bf=\textbf\def\PYG@tc##1{\textcolor[rgb]{0.33,0.33,0.33}{##1}}}
\expandafter\def\csname PYG@tok@ne\endcsname{\def\PYG@tc##1{\textcolor[rgb]{0.00,0.44,0.13}{##1}}}
\expandafter\def\csname PYG@tok@nf\endcsname{\def\PYG@tc##1{\textcolor[rgb]{0.02,0.16,0.49}{##1}}}
\expandafter\def\csname PYG@tok@si\endcsname{\let\PYG@it=\textit\def\PYG@tc##1{\textcolor[rgb]{0.44,0.63,0.82}{##1}}}
\expandafter\def\csname PYG@tok@s2\endcsname{\def\PYG@tc##1{\textcolor[rgb]{0.25,0.44,0.63}{##1}}}
\expandafter\def\csname PYG@tok@vi\endcsname{\def\PYG@tc##1{\textcolor[rgb]{0.73,0.38,0.84}{##1}}}
\expandafter\def\csname PYG@tok@nt\endcsname{\let\PYG@bf=\textbf\def\PYG@tc##1{\textcolor[rgb]{0.02,0.16,0.45}{##1}}}
\expandafter\def\csname PYG@tok@nv\endcsname{\def\PYG@tc##1{\textcolor[rgb]{0.73,0.38,0.84}{##1}}}
\expandafter\def\csname PYG@tok@s1\endcsname{\def\PYG@tc##1{\textcolor[rgb]{0.25,0.44,0.63}{##1}}}
\expandafter\def\csname PYG@tok@gp\endcsname{\let\PYG@bf=\textbf\def\PYG@tc##1{\textcolor[rgb]{0.78,0.36,0.04}{##1}}}
\expandafter\def\csname PYG@tok@sh\endcsname{\def\PYG@tc##1{\textcolor[rgb]{0.25,0.44,0.63}{##1}}}
\expandafter\def\csname PYG@tok@ow\endcsname{\let\PYG@bf=\textbf\def\PYG@tc##1{\textcolor[rgb]{0.00,0.44,0.13}{##1}}}
\expandafter\def\csname PYG@tok@sx\endcsname{\def\PYG@tc##1{\textcolor[rgb]{0.78,0.36,0.04}{##1}}}
\expandafter\def\csname PYG@tok@bp\endcsname{\def\PYG@tc##1{\textcolor[rgb]{0.00,0.44,0.13}{##1}}}
\expandafter\def\csname PYG@tok@c1\endcsname{\let\PYG@it=\textit\def\PYG@tc##1{\textcolor[rgb]{0.25,0.50,0.56}{##1}}}
\expandafter\def\csname PYG@tok@kc\endcsname{\let\PYG@bf=\textbf\def\PYG@tc##1{\textcolor[rgb]{0.00,0.44,0.13}{##1}}}
\expandafter\def\csname PYG@tok@c\endcsname{\let\PYG@it=\textit\def\PYG@tc##1{\textcolor[rgb]{0.25,0.50,0.56}{##1}}}
\expandafter\def\csname PYG@tok@mf\endcsname{\def\PYG@tc##1{\textcolor[rgb]{0.13,0.50,0.31}{##1}}}
\expandafter\def\csname PYG@tok@err\endcsname{\def\PYG@bc##1{\setlength{\fboxsep}{0pt}\fcolorbox[rgb]{1.00,0.00,0.00}{1,1,1}{\strut ##1}}}
\expandafter\def\csname PYG@tok@mb\endcsname{\def\PYG@tc##1{\textcolor[rgb]{0.13,0.50,0.31}{##1}}}
\expandafter\def\csname PYG@tok@ss\endcsname{\def\PYG@tc##1{\textcolor[rgb]{0.32,0.47,0.09}{##1}}}
\expandafter\def\csname PYG@tok@sr\endcsname{\def\PYG@tc##1{\textcolor[rgb]{0.14,0.33,0.53}{##1}}}
\expandafter\def\csname PYG@tok@mo\endcsname{\def\PYG@tc##1{\textcolor[rgb]{0.13,0.50,0.31}{##1}}}
\expandafter\def\csname PYG@tok@kd\endcsname{\let\PYG@bf=\textbf\def\PYG@tc##1{\textcolor[rgb]{0.00,0.44,0.13}{##1}}}
\expandafter\def\csname PYG@tok@mi\endcsname{\def\PYG@tc##1{\textcolor[rgb]{0.13,0.50,0.31}{##1}}}
\expandafter\def\csname PYG@tok@kn\endcsname{\let\PYG@bf=\textbf\def\PYG@tc##1{\textcolor[rgb]{0.00,0.44,0.13}{##1}}}
\expandafter\def\csname PYG@tok@o\endcsname{\def\PYG@tc##1{\textcolor[rgb]{0.40,0.40,0.40}{##1}}}
\expandafter\def\csname PYG@tok@kr\endcsname{\let\PYG@bf=\textbf\def\PYG@tc##1{\textcolor[rgb]{0.00,0.44,0.13}{##1}}}
\expandafter\def\csname PYG@tok@s\endcsname{\def\PYG@tc##1{\textcolor[rgb]{0.25,0.44,0.63}{##1}}}
\expandafter\def\csname PYG@tok@kp\endcsname{\def\PYG@tc##1{\textcolor[rgb]{0.00,0.44,0.13}{##1}}}
\expandafter\def\csname PYG@tok@w\endcsname{\def\PYG@tc##1{\textcolor[rgb]{0.73,0.73,0.73}{##1}}}
\expandafter\def\csname PYG@tok@kt\endcsname{\def\PYG@tc##1{\textcolor[rgb]{0.56,0.13,0.00}{##1}}}
\expandafter\def\csname PYG@tok@sc\endcsname{\def\PYG@tc##1{\textcolor[rgb]{0.25,0.44,0.63}{##1}}}
\expandafter\def\csname PYG@tok@sb\endcsname{\def\PYG@tc##1{\textcolor[rgb]{0.25,0.44,0.63}{##1}}}
\expandafter\def\csname PYG@tok@k\endcsname{\let\PYG@bf=\textbf\def\PYG@tc##1{\textcolor[rgb]{0.00,0.44,0.13}{##1}}}
\expandafter\def\csname PYG@tok@se\endcsname{\let\PYG@bf=\textbf\def\PYG@tc##1{\textcolor[rgb]{0.25,0.44,0.63}{##1}}}
\expandafter\def\csname PYG@tok@sd\endcsname{\let\PYG@it=\textit\def\PYG@tc##1{\textcolor[rgb]{0.25,0.44,0.63}{##1}}}

\def\PYGZbs{\char`\\}
\def\PYGZus{\char`\_}
\def\PYGZob{\char`\{}
\def\PYGZcb{\char`\}}
\def\PYGZca{\char`\^}
\def\PYGZam{\char`\&}
\def\PYGZlt{\char`\<}
\def\PYGZgt{\char`\>}
\def\PYGZsh{\char`\#}
\def\PYGZpc{\char`\%}
\def\PYGZdl{\char`\$}
\def\PYGZhy{\char`\-}
\def\PYGZsq{\char`\'}
\def\PYGZdq{\char`\"}
\def\PYGZti{\char`\~}
% for compatibility with earlier versions
\def\PYGZat{@}
\def\PYGZlb{[}
\def\PYGZrb{]}
\makeatother

\renewcommand\PYGZsq{\textquotesingle}

\begin{document}

\maketitle
\tableofcontents
\phantomsection\label{index::doc}


Contents:


\chapter{astroobs package}
\label{astroobs:welcome-to-astroobs-s-documentation}\label{astroobs:astroobs-package}\label{astroobs::doc}

\section{Submodules}
\label{astroobs:submodules}

\section{astroobs.Moon module}
\label{astroobs:module-astroobs.Moon}\label{astroobs:astroobs-moon-module}\index{astroobs.Moon (module)}\index{Moon (class in astroobs.Moon)}

\begin{fulllineitems}
\phantomsection\label{astroobs:astroobs.Moon.Moon}\pysiglinewithargsret{\strong{class }\code{astroobs.Moon.}\bfcode{Moon}}{\emph{obs=None}, \emph{input\_epoch=`2000'}, \emph{**kwargs}}{}
Bases: \code{astroobs.Target.Target}

Initialises the Moon. Optionaly, processes the Moon for the observatory and date given (refer to \code{Moon.process()}).
\begin{description}
\item[{Args:}] \leavevmode\begin{itemize}
\item {} 
obs (\code{Observatory}) {[}optional{]}: the observatory for which to process the Moon

\end{itemize}

\item[{Kwargs:}] \leavevmode\begin{itemize}
\item {} 
raiseError (bool): if \code{True}, errors will be raised; if \code{False}, they will be printed. Default is \code{False}

\end{itemize}

\item[{Raises:}] \leavevmode
N/A

\end{description}
\index{dec (astroobs.Moon.Moon attribute)}

\begin{fulllineitems}
\phantomsection\label{astroobs:astroobs.Moon.Moon.dec}\pysigline{\bfcode{dec}}
The declination of the Moon, displayed as tuple of np.array (+/-dd, mm, ss)

\end{fulllineitems}

\index{decStr (astroobs.Moon.Moon attribute)}

\begin{fulllineitems}
\phantomsection\label{astroobs:astroobs.Moon.Moon.decStr}\pysigline{\bfcode{decStr}}
A pretty printable version of the mean of the declination of the moon

\end{fulllineitems}

\index{plot() (astroobs.Moon.Moon method)}

\begin{fulllineitems}
\phantomsection\label{astroobs:astroobs.Moon.Moon.plot}\pysiglinewithargsret{\bfcode{plot}}{\emph{obs}, \emph{y='alt'}, \emph{**kwargs}}{}
Plots the y-parameter vs time diagram for the moon at the given observatory and date
\begin{description}
\item[{Args:}] \leavevmode\begin{itemize}
\item {} 
obs (\code{Observatory}): the observatory for which to plot the moon

\end{itemize}

\item[{Kwargs:}] \leavevmode\begin{itemize}
\item {} 
See class constructor

\item {} 
See \code{Observatory.plot()}

\item {} 
See \code{Target.plot()}

\end{itemize}

\item[{Raises:}] \leavevmode
N/A

\end{description}

\end{fulllineitems}

\index{polar() (astroobs.Moon.Moon method)}

\begin{fulllineitems}
\phantomsection\label{astroobs:astroobs.Moon.Moon.polar}\pysiglinewithargsret{\bfcode{polar}}{\emph{obs}, \emph{**kwargs}}{}
Plots the y-parameter vs time diagram for the moon at the given observatory and date
\begin{description}
\item[{Args:}] \leavevmode\begin{itemize}
\item {} 
obs (\code{Observatory}): the observatory for which to plot the moon

\end{itemize}

\item[{Kwargs:}] \leavevmode\begin{itemize}
\item {} 
See class constructor

\item {} 
See \code{Observatory.plot()}

\item {} 
See \code{Target.plot()}

\end{itemize}

\item[{Raises:}] \leavevmode
N/A

\end{description}

\end{fulllineitems}

\index{process() (astroobs.Moon.Moon method)}

\begin{fulllineitems}
\phantomsection\label{astroobs:astroobs.Moon.Moon.process}\pysiglinewithargsret{\bfcode{process}}{\emph{obs}, \emph{**kwargs}}{}
Processes the moon for the given observatory and date.
\begin{description}
\item[{Args:}] \leavevmode\begin{itemize}
\item {} 
obs (\code{Observatory}): the observatory for which to process the moon

\end{itemize}

\item[{Kwargs:}] \leavevmode
See class constructor

\item[{Raises:}] \leavevmode
N/A

\item[{Creates vector attributes:}] \leavevmode\begin{itemize}
\item {} 
\code{airmass}: the airmass of the moon

\item {} 
\code{ha}: the hour angle of the moon (degrees)

\item {} 
\code{alt}: the altitude of the moon (degrees - horizon is 0)

\item {} 
\code{az}: the azimuth of the moon (degrees)

\item {} 
\code{ra}: the right ascension of the moon, see \code{Moon.ra()}

\item {} 
\code{dec}: the declination of the moon, see \code{Moon.dec()}

\end{itemize}

\end{description}

\begin{notice}{note}{Note:}\begin{itemize}
\item {} 
All previous attributes are vectors related to the time vector of the observatory used for processing: \code{obs.dates}

\end{itemize}
\end{notice}
\begin{description}
\item[{Other attributes:}] \leavevmode\begin{itemize}
\item {} 
\code{rise\_time}, \code{rise\_az}: the time (ephem.Date) and the azimuth (degree) of the rise of the moon

\item {} 
\code{set\_time}, \code{set\_az}: the time (ephem.Date) and the azimuth (degree) of the setting of the moon

\item {} 
\code{transit\_time}, \code{transit\_az}: the time (ephem.Date) and the azimuth (degree) of the transit of the moon

\end{itemize}

\end{description}

\begin{notice}{warning}{Warning:}\begin{itemize}
\item {} 
it can occur that the moon does not rise or set for an observatory/date combination. In that case, the corresponding attributes will be set to \code{None}, i.e. \code{set\_time}, \code{set\_az}, \code{rise\_time}, \code{rise\_az}. In that case, an additional parameter is added to the Moon object: \code{Moon.alwaysUp} which is \code{True} if the Moon never sets and \code{False} if it never rises above the horizon.

\end{itemize}
\end{notice}

\end{fulllineitems}

\index{ra (astroobs.Moon.Moon attribute)}

\begin{fulllineitems}
\phantomsection\label{astroobs:astroobs.Moon.Moon.ra}\pysigline{\bfcode{ra}}
The right ascension of the Moon, displayed as tuple of np.array (hh, mm, ss)

\end{fulllineitems}

\index{raStr (astroobs.Moon.Moon attribute)}

\begin{fulllineitems}
\phantomsection\label{astroobs:astroobs.Moon.Moon.raStr}\pysigline{\bfcode{raStr}}
A pretty printable version of the mean of the right ascension of the moon

\end{fulllineitems}


\end{fulllineitems}



\section{astroobs.Observation module}
\label{astroobs:astroobs-observation-module}\label{astroobs:module-astroobs.Observation}\index{astroobs.Observation (module)}\index{Observation (class in astroobs.Observation)}

\begin{fulllineitems}
\phantomsection\label{astroobs:astroobs.Observation.Observation}\pysiglinewithargsret{\strong{class }\code{astroobs.Observation.}\bfcode{Observation}}{\emph{obs}, \emph{long=None}, \emph{lat=None}, \emph{elevation=None}, \emph{timezone=None}, \emph{temp=None}, \emph{pressure=None}, \emph{moonAvoidRadius=None}, \emph{local\_date=None}, \emph{ut\_date=None}, \emph{horizon\_obs=None}, \emph{dataFile=None}, \emph{epoch=`2000'}, \emph{**kwargs}}{}
Bases: \code{astroobs.Observatory.Observatory}

Assembles together an \code{Observatory} (including itself the \code{Moon} target), and a list of \code{Target}.
\begin{description}
\item[{For use and docs refer to:}] \leavevmode\begin{itemize}
\item {} 
\code{add\_target()} to add a target to the list

\item {} 
\code{rem\_target()} to remove one

\item {} 
\code{change\_obs()} to change the observatory

\item {} 
\code{change\_date()} to change the date of observation

\end{itemize}

\item[{Kwargs:}] \leavevmode\begin{itemize}
\item {} 
raiseError (bool): if \code{True}, errors will be raised; if \code{False}, they will be printed. Default is \code{False}

\item {} 
fig: TBD

\end{itemize}

\item[{Raises:}] \leavevmode
See \code{Observatory}

\end{description}

\begin{notice}{warning}{Warning:}\begin{itemize}
\item {} 
it can occur that the Sun, the Moon or a target does not rise or set for an observatory/date combination. In that case, the corresponding attributes will be set to \code{None}

\end{itemize}
\end{notice}

\begin{Verbatim}[commandchars=\\\{\}]
\PYG{g+gp}{\PYGZgt{}\PYGZgt{}\PYGZgt{} }\PYG{k+kn}{import} \PYG{n+nn}{astroobs.obs} \PYG{k+kn}{as} \PYG{n+nn}{obs}
\PYG{g+gp}{\PYGZgt{}\PYGZgt{}\PYGZgt{} }\PYG{n}{o} \PYG{o}{=} \PYG{n}{obs}\PYG{o}{.}\PYG{n}{Observation}\PYG{p}{(}\PYG{l+s}{\PYGZsq{}}\PYG{l+s}{ohp}\PYG{l+s}{\PYGZsq{}}\PYG{p}{,} \PYG{n}{local\PYGZus{}date}\PYG{o}{=}\PYG{p}{(}\PYG{l+m+mi}{2015}\PYG{p}{,}\PYG{l+m+mi}{3}\PYG{p}{,}\PYG{l+m+mi}{31}\PYG{p}{,}\PYG{l+m+mi}{23}\PYG{p}{,}\PYG{l+m+mi}{59}\PYG{p}{,}\PYG{l+m+mi}{59}\PYG{p}{)}\PYG{p}{)}
\PYG{g+gp}{\PYGZgt{}\PYGZgt{}\PYGZgt{} }\PYG{n}{o}
\PYG{g+go}{Observation at Observatoire de Haute Provence on 2015/6/21\PYGZhy{}22. 0 targets.}
\PYG{g+go}{    Moon phase: 89.2\PYGZpc{}}
\PYG{g+gp}{\PYGZgt{}\PYGZgt{}\PYGZgt{} }\PYG{n}{o}\PYG{o}{.}\PYG{n}{moon}
\PYG{g+go}{Moon \PYGZhy{} phase: 89.2\PYGZpc{}}
\PYG{g+gp}{\PYGZgt{}\PYGZgt{}\PYGZgt{} }\PYG{k}{print} \PYG{n}{o}\PYG{o}{.}\PYG{n}{sunset}\PYG{p}{,} \PYG{l+s}{\PYGZsq{}}\PYG{l+s}{...}\PYG{l+s}{\PYGZsq{}}\PYG{p}{,} \PYG{n}{o}\PYG{o}{.}\PYG{n}{sunrise}\PYG{p}{,} \PYG{l+s}{\PYGZsq{}}\PYG{l+s}{...}\PYG{l+s}{\PYGZsq{}}\PYG{p}{,} \PYG{n}{o}\PYG{o}{.}\PYG{n}{len\PYGZus{}night}
\PYG{g+go}{2015/3/31 18:08:40 ... 2015/4/1 05:13:09 ... 11.0746939826}
\PYG{g+gp}{\PYGZgt{}\PYGZgt{}\PYGZgt{} }\PYG{k+kn}{import} \PYG{n+nn}{ephem} \PYG{k+kn}{as} \PYG{n+nn}{E}
\PYG{g+gp}{\PYGZgt{}\PYGZgt{}\PYGZgt{} }\PYG{k}{print}\PYG{p}{(}\PYG{n}{E}\PYG{o}{.}\PYG{n}{Date}\PYG{p}{(}\PYG{n}{o}\PYG{o}{.}\PYG{n}{sunsetastro}\PYG{o}{+}\PYG{n}{o}\PYG{o}{.}\PYG{n}{localTimeOffest}\PYG{p}{)}\PYG{p}{,} \PYG{l+s}{\PYGZsq{}}\PYG{l+s}{...}\PYG{l+s}{\PYGZsq{}}\PYG{p}{,} \PYG{n}{E}\PYG{o}{.}\PYG{n}{Date}\PYG{p}{(}
\PYG{g+go}{        o.sunriseastro+o.localTimeOffest), \PYGZsq{}...\PYGZsq{}, o.len\PYGZus{}nightastro)}
\PYG{g+go}{2015/3/31 21:43:28 ... 2015/4/1 05:38:26 ... 7.91603336949}
\PYG{g+gp}{\PYGZgt{}\PYGZgt{}\PYGZgt{} }\PYG{n}{o}\PYG{o}{.}\PYG{n}{add\PYGZus{}target}\PYG{p}{(}\PYG{l+s}{\PYGZsq{}}\PYG{l+s}{vega}\PYG{l+s}{\PYGZsq{}}\PYG{p}{)}
\PYG{g+gp}{\PYGZgt{}\PYGZgt{}\PYGZgt{} }\PYG{n}{o}\PYG{o}{.}\PYG{n}{add\PYGZus{}target}\PYG{p}{(}\PYG{l+s}{\PYGZsq{}}\PYG{l+s}{mystar}\PYG{l+s}{\PYGZsq{}}\PYG{p}{,} \PYG{n}{dec}\PYG{o}{=}\PYG{l+m+mf}{19.1824}\PYG{p}{,} \PYG{n}{ra}\PYG{o}{=}\PYG{l+m+mf}{213.9153}\PYG{p}{)}
\PYG{g+gp}{\PYGZgt{}\PYGZgt{}\PYGZgt{} }\PYG{n}{o}\PYG{o}{.}\PYG{n}{targets}
\PYG{g+go}{[Target: \PYGZsq{}vega\PYGZsq{}, 18h36m56.3s +38°35\PYGZsq{}8.1\PYGZdq{}, O,}
\PYG{g+go}{ Target: \PYGZsq{}mystar\PYGZsq{}, 14h15m39.7s +19°16\PYGZsq{}43.8\PYGZdq{}, O]}
\PYG{g+gp}{\PYGZgt{}\PYGZgt{}\PYGZgt{} }\PYG{k}{print}\PYG{p}{(}\PYG{l+s}{\PYGZdq{}}\PYG{l+s+si}{\PYGZpc{}s}\PYG{l+s}{ mags: }\PYG{l+s}{\PYGZsq{}}\PYG{l+s}{K}\PYG{l+s}{\PYGZsq{}}\PYG{l+s}{: }\PYG{l+s+si}{\PYGZpc{}2.2f}\PYG{l+s}{, }\PYG{l+s}{\PYGZsq{}}\PYG{l+s}{R}\PYG{l+s}{\PYGZsq{}}\PYG{l+s}{: }\PYG{l+s+si}{\PYGZpc{}2.2f}\PYG{l+s}{\PYGZdq{}}\PYG{o}{\PYGZpc{}}\PYG{p}{(}\PYG{n}{o}\PYG{o}{.}\PYG{n}{targets}\PYG{p}{[}\PYG{l+m+mi}{0}\PYG{p}{]}\PYG{o}{.}\PYG{n}{name}\PYG{p}{,}
\PYG{g+go}{        o.targets[0].flux[\PYGZsq{}K\PYGZsq{}], o.targets[0].flux[\PYGZsq{}R\PYGZsq{}]))}
\PYG{g+go}{vega mags: \PYGZsq{}K\PYGZsq{}: 0.13, \PYGZsq{}R\PYGZsq{}: 0.07}
\end{Verbatim}
\index{add\_target() (astroobs.Observation.Observation method)}

\begin{fulllineitems}
\phantomsection\label{astroobs:astroobs.Observation.Observation.add_target}\pysiglinewithargsret{\bfcode{add\_target}}{\emph{tgt}, \emph{ra=None}, \emph{dec=None}, \emph{name='`}, \emph{**kwargs}}{}
Adds a target to the observation list
\begin{description}
\item[{Args:}] \leavevmode\begin{itemize}
\item {} 
tgt (see below): the index of the target in the \code{Observation.targets} list

\item {} 
ra (`hh:mm:ss.s' or decimal degree) {[}optional{]}: the right ascension of the target to add to the observation list. See below

\item {} 
dec (`+/-dd:mm:ss.s' or decimal degree) {[}optional{]}: the declination of the target to add to the observation list. See below

\item {} 
name (string) {[}optional{]}: the name of the target to add to the observation list. See below

\end{itemize}

\item[{\code{tgt} arg can be:}] \leavevmode\begin{itemize}
\item {} 
a \code{Target} instance: all other parameters are ignored

\item {} 
a target name (string): if \code{ra} and \code{dec} are not \code{None}, the target is added with the provided coordinates; if \code{None}, a SIMBAD search is performed on \code{tgt}. \code{name} is ignored

\item {} 
a ra-dec string (`hh:mm:ss.s +/-dd:mm:ss.s'): in that case, \code{ra} and \code{dec} will be ignored and \code{name} will be the name of the target

\end{itemize}

\item[{Kwargs:}] \leavevmode
See class constructor

\item[{Raises:}] \leavevmode\begin{itemize}
\item {} 
ValueError: if ra-dec formating was not understood

\end{itemize}

\end{description}

\begin{notice}{note}{Note:}\begin{itemize}
\item {} 
Automatically processes the target for the given observatory and date

\end{itemize}
\end{notice}

\begin{Verbatim}[commandchars=\\\{\}]
\PYG{g+gp}{\PYGZgt{}\PYGZgt{}\PYGZgt{} }\PYG{k+kn}{import} \PYG{n+nn}{astroobs.obs} \PYG{k+kn}{as} \PYG{n+nn}{obs}
\PYG{g+gp}{\PYGZgt{}\PYGZgt{}\PYGZgt{} }\PYG{n}{o} \PYG{o}{=} \PYG{n}{obs}\PYG{o}{.}\PYG{n}{Observation}\PYG{p}{(}\PYG{l+s}{\PYGZsq{}}\PYG{l+s}{ohp}\PYG{l+s}{\PYGZsq{}}\PYG{p}{,} \PYG{n}{local\PYGZus{}date}\PYG{o}{=}\PYG{p}{(}\PYG{l+m+mi}{2015}\PYG{p}{,}\PYG{l+m+mi}{3}\PYG{p}{,}\PYG{l+m+mi}{31}\PYG{p}{,}\PYG{l+m+mi}{23}\PYG{p}{,}\PYG{l+m+mi}{59}\PYG{p}{,}\PYG{l+m+mi}{59}\PYG{p}{)}\PYG{p}{)}
\PYG{g+gp}{\PYGZgt{}\PYGZgt{}\PYGZgt{} }\PYG{n}{arc} \PYG{o}{=} \PYG{n}{obs}\PYG{o}{.}\PYG{n}{TargetSIMBAD}\PYG{p}{(}\PYG{l+s}{\PYGZsq{}}\PYG{l+s}{arcturus}\PYG{l+s}{\PYGZsq{}}\PYG{p}{)}
\PYG{g+gp}{\PYGZgt{}\PYGZgt{}\PYGZgt{} }\PYG{n}{o}\PYG{o}{.}\PYG{n}{add\PYGZus{}target}\PYG{p}{(}\PYG{n}{arc}\PYG{p}{)}
\PYG{g+gp}{\PYGZgt{}\PYGZgt{}\PYGZgt{} }\PYG{n}{o}\PYG{o}{.}\PYG{n}{add\PYGZus{}target}\PYG{p}{(}\PYG{l+s}{\PYGZsq{}}\PYG{l+s}{arcturus}\PYG{l+s}{\PYGZsq{}}\PYG{p}{)}
\PYG{g+gp}{\PYGZgt{}\PYGZgt{}\PYGZgt{} }\PYG{n}{o}\PYG{o}{.}\PYG{n}{add\PYGZus{}target}\PYG{p}{(}\PYG{l+s}{\PYGZsq{}}\PYG{l+s}{arcturusILoveYou}\PYG{l+s}{\PYGZsq{}}\PYG{p}{,} \PYG{n}{dec}\PYG{o}{=}\PYG{l+m+mf}{19.1824}\PYG{p}{,} \PYG{n}{ra}\PYG{o}{=}\PYG{l+m+mf}{213.9153}\PYG{p}{)}
\PYG{g+gp}{\PYGZgt{}\PYGZgt{}\PYGZgt{} }\PYG{n}{o}\PYG{o}{.}\PYG{n}{add\PYGZus{}target}\PYG{p}{(}\PYG{l+s}{\PYGZsq{}}\PYG{l+s}{14:15:39.67 +10:10:56.67}\PYG{l+s}{\PYGZsq{}}\PYG{p}{,} \PYG{n}{name}\PYG{o}{=}\PYG{l+s}{\PYGZsq{}}\PYG{l+s}{arcturus}\PYG{l+s}{\PYGZsq{}}\PYG{p}{)}
\PYG{g+gp}{\PYGZgt{}\PYGZgt{}\PYGZgt{} }\PYG{n}{o}\PYG{o}{.}\PYG{n}{targets} 
\PYG{g+go}{[Target: \PYGZsq{}arcturus\PYGZsq{}, 14h15m39.7s +19°16\PYGZsq{}43.8\PYGZdq{}, O,}
\PYG{g+go}{ Target: \PYGZsq{}arcturus\PYGZsq{}, 14h15m39.7s +19°16\PYGZsq{}43.8\PYGZdq{}, O,}
\PYG{g+go}{ Target: \PYGZsq{}arcturus\PYGZsq{}, 14h15m39.7s +10°40\PYGZsq{}43.8\PYGZdq{}, O,}
\PYG{g+go}{ Target: \PYGZsq{}arcturus\PYGZsq{}, 14h15m39.7s +19°16\PYGZsq{}43.8\PYGZdq{}, O]}
\end{Verbatim}

\end{fulllineitems}

\index{change\_date() (astroobs.Observation.Observation method)}

\begin{fulllineitems}
\phantomsection\label{astroobs:astroobs.Observation.Observation.change_date}\pysiglinewithargsret{\bfcode{change\_date}}{\emph{ut\_date=None}, \emph{local\_date=None}, \emph{recalcAll=False}, \emph{**kwargs}}{}
Changes the date of the observation and optionaly re-processes targets for the same observatory and new date
\begin{description}
\item[{Args:}] \leavevmode\begin{itemize}
\item {} 
ut\_date: Refer to \code{Observatory.upd\_date()}

\item {} 
local\_date: Refer to \code{Observatory.upd\_date()}

\item {} 
recalcAll (bool or None) {[}optional{]}: if \code{False} (default): only targets selected for observation are re-processed, if \code{True}: all targets are re-processed, if \code{None}: no re-process

\end{itemize}

\item[{Kwargs:}] \leavevmode
See class constructor

\item[{Raises:}] \leavevmode\begin{itemize}
\item {} 
KeyError: if the twilight keyword is unknown

\item {} 
Exception: if the observatory object has no date

\end{itemize}

\end{description}

\end{fulllineitems}

\index{change\_obs() (astroobs.Observation.Observation method)}

\begin{fulllineitems}
\phantomsection\label{astroobs:astroobs.Observation.Observation.change_obs}\pysiglinewithargsret{\bfcode{change\_obs}}{\emph{obs}, \emph{long=None}, \emph{lat=None}, \emph{elevation=None}, \emph{timezone=None}, \emph{temp=None}, \emph{pressure=None}, \emph{moonAvoidRadius=None}, \emph{horizon\_obs=None}, \emph{dataFile=None}, \emph{recalcAll=False}, \emph{**kwargs}}{}
Changes the observatory and optionaly re-processes all target for the new observatory and same date
\begin{description}
\item[{Args:}] \leavevmode\begin{itemize}
\item {} 
recalcAll (bool or None) {[}optional{]}: if \code{False} (default): only targets selected for observation are re-processed, if \code{True}: all targets are re-processed, if \code{None}: no re-process

\end{itemize}

\item[{Kwargs:}] \leavevmode
See class constructor

\end{description}

\begin{notice}{note}{Note:}\begin{itemize}
\item {} 
Refer to \code{ObservatoryList.add()} for details on other input parameters

\end{itemize}
\end{notice}

\end{fulllineitems}

\index{plot() (astroobs.Observation.Observation method)}

\begin{fulllineitems}
\phantomsection\label{astroobs:astroobs.Observation.Observation.plot}\pysiglinewithargsret{\bfcode{plot}}{\emph{y='alt'}, \emph{**kwargs}}{}
Plots the y-parameter vs time diagram for the target at the given observatory and date
\begin{description}
\item[{Kwargs:}] \leavevmode\begin{itemize}
\item {} 
See class constructor

\item {} 
See \code{Observatory.plot()}

\item {} 
moon (bool): if \code{True}, adds the moon to the graph, default is \code{True}

\item {} 
autocolor (bool): if \code{True}, sets curves-colors automatically, default is \code{True}

\end{itemize}

\item[{Raises:}] \leavevmode
N/A

\end{description}

\end{fulllineitems}

\index{polar() (astroobs.Observation.Observation method)}

\begin{fulllineitems}
\phantomsection\label{astroobs:astroobs.Observation.Observation.polar}\pysiglinewithargsret{\bfcode{polar}}{\emph{**kwargs}}{}
Plots the sky-view diagram for the target at the given observatory and date
\begin{description}
\item[{Kwargs:}] \leavevmode\begin{itemize}
\item {} 
See class constructor

\item {} 
See \code{Observatory.plot()}

\item {} 
moon (bool): if \code{True}, adds the moon to the graph, default is \code{True}

\item {} 
autocolor (bool): if \code{True}, sets curves-colors automatically, default is \code{True}

\end{itemize}

\item[{Raises:}] \leavevmode
N/A

\end{description}

\end{fulllineitems}

\index{rem\_target() (astroobs.Observation.Observation method)}

\begin{fulllineitems}
\phantomsection\label{astroobs:astroobs.Observation.Observation.rem_target}\pysiglinewithargsret{\bfcode{rem\_target}}{\emph{tgt}, \emph{**kwargs}}{}
Removes a target from the observation list
\begin{description}
\item[{Args:}] \leavevmode\begin{itemize}
\item {} 
tgt (int): the index of the target in the \code{Observation.targets} list

\end{itemize}

\item[{Kwargs:}] \leavevmode
See class constructor

\item[{Raises:}] \leavevmode
N/A

\end{description}

\end{fulllineitems}

\index{targets (astroobs.Observation.Observation attribute)}

\begin{fulllineitems}
\phantomsection\label{astroobs:astroobs.Observation.Observation.targets}\pysigline{\bfcode{targets}}
Shows the list of targets recorded into the Observation

\end{fulllineitems}

\index{tick() (astroobs.Observation.Observation method)}

\begin{fulllineitems}
\phantomsection\label{astroobs:astroobs.Observation.Observation.tick}\pysiglinewithargsret{\bfcode{tick}}{\emph{tgt}, \emph{forceTo=None}, \emph{**kwargs}}{}
Changes the ticked property of a target (whether it is selected for observation)
\begin{description}
\item[{Args:}] \leavevmode\begin{itemize}
\item {} 
tgt (int): the index of the target in the \code{Observation.targets} list

\item {} 
forceTo (bool) {[}optional{]}: if \code{True}, selects the target for observation, if \code{False}, unselects it, if \code{None}, the value of the selection is inverted

\end{itemize}

\item[{Kwargs:}] \leavevmode
See class constructor

\item[{Raises:}] \leavevmode
N/A

\end{description}

\begin{notice}{note}{Note:}\begin{itemize}
\item {} 
Automatically reprocesses the target for the given observatory and date if it is selected for observation

\end{itemize}
\end{notice}

\begin{Verbatim}[commandchars=\\\{\}]
\PYG{g+gp}{\PYGZgt{}\PYGZgt{}\PYGZgt{} }\PYG{k+kn}{import} \PYG{n+nn}{astroobs.obs} \PYG{k+kn}{as} \PYG{n+nn}{obs}
\PYG{g+gp}{\PYGZgt{}\PYGZgt{}\PYGZgt{} }\PYG{n}{o} \PYG{o}{=} \PYG{n}{obs}\PYG{o}{.}\PYG{n}{Observation}\PYG{p}{(}\PYG{l+s}{\PYGZsq{}}\PYG{l+s}{ohp}\PYG{l+s}{\PYGZsq{}}\PYG{p}{,} \PYG{n}{local\PYGZus{}date}\PYG{o}{=}\PYG{p}{(}\PYG{l+m+mi}{2015}\PYG{p}{,}\PYG{l+m+mi}{3}\PYG{p}{,}\PYG{l+m+mi}{31}\PYG{p}{,}\PYG{l+m+mi}{23}\PYG{p}{,}\PYG{l+m+mi}{59}\PYG{p}{,}\PYG{l+m+mi}{59}\PYG{p}{)}\PYG{p}{)}
\PYG{g+gp}{\PYGZgt{}\PYGZgt{}\PYGZgt{} }\PYG{n}{o}\PYG{o}{.}\PYG{n}{add\PYGZus{}target}\PYG{p}{(}\PYG{l+s}{\PYGZsq{}}\PYG{l+s}{arcturus}\PYG{l+s}{\PYGZsq{}}\PYG{p}{)}
\PYG{g+gp}{\PYGZgt{}\PYGZgt{}\PYGZgt{} }\PYG{n}{o}\PYG{o}{.}\PYG{n}{targets}
\PYG{g+go}{[Target: \PYGZsq{}arcturus\PYGZsq{}, 14h15m39.7s +19°16\PYGZsq{}43.8\PYGZdq{}, O]}
\PYG{g+gp}{\PYGZgt{}\PYGZgt{}\PYGZgt{} }\PYG{n}{o}\PYG{o}{.}\PYG{n}{tick}\PYG{p}{(}\PYG{l+m+mi}{4}\PYG{p}{)}
\PYG{g+gp}{\PYGZgt{}\PYGZgt{}\PYGZgt{} }\PYG{n}{o}\PYG{o}{.}\PYG{n}{targets}
\PYG{g+go}{[Target: \PYGZsq{}arcturus\PYGZsq{}, 14h15m39.7s +19°16\PYGZsq{}43.8\PYGZdq{}, \PYGZhy{}]}
\end{Verbatim}

\end{fulllineitems}

\index{ticked (astroobs.Observation.Observation attribute)}

\begin{fulllineitems}
\phantomsection\label{astroobs:astroobs.Observation.Observation.ticked}\pysigline{\bfcode{ticked}}
Shows whether the target was select for observation

\end{fulllineitems}


\end{fulllineitems}



\section{astroobs.Observatory module}
\label{astroobs:astroobs-observatory-module}\label{astroobs:module-astroobs.Observatory}\index{astroobs.Observatory (module)}\index{Observatory (class in astroobs.Observatory)}

\begin{fulllineitems}
\phantomsection\label{astroobs:astroobs.Observatory.Observatory}\pysiglinewithargsret{\strong{class }\code{astroobs.Observatory.}\bfcode{Observatory}}{\emph{obs}, \emph{long=None}, \emph{lat=None}, \emph{elevation=None}, \emph{timezone=None}, \emph{temp=None}, \emph{pressure=None}, \emph{moonAvoidRadius=None}, \emph{local\_date=None}, \emph{ut\_date=None}, \emph{horizon\_obs=None}, \emph{dataFile=None}, \emph{epoch=`2000'}, \emph{**kwargs}}{}
Bases: \code{ephem.Observer}, \code{object}

Defines an observatory from which the ephemeris of the twilights or a night-sky target are processed. The \emph{night-time} is base on the given date. It ends at the next sunrise and starts at the sunset preceeding this next sunrise.
\begin{description}
\item[{Args:}] \leavevmode\begin{itemize}
\item {} 
obs (str): id of the observatory to pick from the observatories database OR the name of the custom observatory (in that case, \code{long}, \code{lat}, \code{elevation}, \code{timezone} must also be given, \code{temp}, \code{pressure}, \code{moonAvoidRadius} are optional)

\item {} 
local\_date (see below): the date of observation in local time

\item {} 
ut\_date (see below): the date of observation in UT time

\item {} 
horizon\_obs (float - degrees): minimum altitude at which a target can be observed, default is 30 degrees altitude

\item {} 
epoch (str): the `YYYY' year in which all ra-dec coordinates are converted

\end{itemize}

\item[{Kwargs:}] \leavevmode\begin{itemize}
\item {} 
raiseError (bool): if \code{True}, errors will be raised; if \code{False}, they will be printed. Default is \code{False}

\item {} 
fig: TBD

\end{itemize}

\item[{Raises:}] \leavevmode\begin{itemize}
\item {} 
NameError: if a mandatory input parameter is missing

\item {} 
KeyError: if the observatory ID does not exist

\item {} 
KeyError: if the twilight keyword is unknown

\item {} 
Exception: if the observatory object has no date

\end{itemize}

\end{description}

\begin{notice}{note}{Note:}\begin{itemize}
\item {} 
For details on \code{local\_date} and \code{ut\_date}, refer to \code{Observatory.upd\_date()}

\item {} 
For details on other input parameters, refer to \code{ObservatoryList.add()}

\item {} 
The \code{Observatory} automatically creates and manages a \code{Moon} target under \code{moon} attribute

\item {} 
If \code{obs} is the id of an observatory to pick in the database, the user can still provide \code{temp}, \code{pressure}, \code{moonAvoidRadius} attributes which will override the database default values

\item {} 
\code{horizon} attribute is in radian

\end{itemize}
\end{notice}
\begin{description}
\item[{Main attributes:}] \leavevmode\begin{itemize}
\item {} 
\code{localnight}: gives the local midnight time in local time (YYYY, MM, DD, 23, 59, 59)

\item {} 
\code{date}: gives the local midnight time in UT time

\item {} 
\code{dates}: is a vector of Dublin Julian Dates. Refer to \code{process\_obs()}

\item {} 
\code{lst}: the local sidereal time corresponding to each \code{dates} element

\item {} 
\code{localTimeOffest}: gives the shift in days between UT and local time: local=UT+localTimeOffest

\item {} 
\code{moon}: points to the \code{Moon} target processed for the given observatory and date

\end{itemize}

\item[{Twilight attributes:}] \leavevmode\begin{itemize}
\item {} 
For the next three attributes, \code{XXX} shall be replaced by \{`' (blank), `civil', `nautical', `astro'\} for, respectively, horizon, -6, -12, and -18 degrees altitude

\item {} 
\code{sunriseXXX}: gives the sunrise time for different twilights, in Dublin Julian Dates. e.g.: \code{observatory.sunrise}

\item {} 
\code{sunsetXXX}: gives the sunset time for different twilights, in Dublin Julian Dates. e.g.: \code{observatory.sunsetcivil}

\item {} 
\code{len\_nightXXX}: gives the night duration for different twilights (between corresponding sunset and sunrise), in hours. e.g.: \code{observatory.len\_nightnautical}

\end{itemize}

\end{description}

\begin{notice}{warning}{Warning:}\begin{itemize}
\item {} 
it can occur that the Sun, the Moon or a target does not rise or set for an observatory/date combination. In that case, the corresponding attributes will be set to \code{None}

\end{itemize}
\end{notice}

\begin{Verbatim}[commandchars=\\\{\}]
\PYG{g+gp}{\PYGZgt{}\PYGZgt{}\PYGZgt{} }\PYG{k+kn}{import} \PYG{n+nn}{astroobs.obs} \PYG{k+kn}{as} \PYG{n+nn}{obs}
\PYG{g+gp}{\PYGZgt{}\PYGZgt{}\PYGZgt{} }\PYG{n}{o} \PYG{o}{=} \PYG{n}{obs}\PYG{o}{.}\PYG{n}{Observatory}\PYG{p}{(}\PYG{l+s}{\PYGZsq{}}\PYG{l+s}{ohp}\PYG{l+s}{\PYGZsq{}}\PYG{p}{,} \PYG{n}{local\PYGZus{}date}\PYG{o}{=}\PYG{p}{(}\PYG{l+m+mi}{2015}\PYG{p}{,}\PYG{l+m+mi}{3}\PYG{p}{,}\PYG{l+m+mi}{31}\PYG{p}{,}\PYG{l+m+mi}{23}\PYG{p}{,}\PYG{l+m+mi}{59}\PYG{p}{,}\PYG{l+m+mi}{59}\PYG{p}{)}\PYG{p}{)}
\PYG{g+gp}{\PYGZgt{}\PYGZgt{}\PYGZgt{} }\PYG{n}{o}
\PYG{g+go}{\PYGZlt{}ephem.Observer date=\PYGZsq{}2015/3/31 21:59:59\PYGZsq{} epoch=\PYGZsq{}2000/1/1 12:00:00\PYGZsq{}}
\PYG{g+go}{lon=5:42:48.0 lat=43:55:51.0 elevation=650.0m horizon=\PYGZhy{}0:49:04.8}
\PYG{g+go}{temp=15.0C pressure=1010.0mBar\PYGZgt{}}
\PYG{g+gp}{\PYGZgt{}\PYGZgt{}\PYGZgt{} }\PYG{n}{o}\PYG{o}{.}\PYG{n}{moon}
\PYG{g+go}{Moon \PYGZhy{} phase: 89.2\PYGZpc{}}
\PYG{g+gp}{\PYGZgt{}\PYGZgt{}\PYGZgt{} }\PYG{k}{print}\PYG{p}{(}\PYG{n}{o}\PYG{o}{.}\PYG{n}{sunset}\PYG{p}{,} \PYG{l+s}{\PYGZsq{}}\PYG{l+s}{...}\PYG{l+s}{\PYGZsq{}}\PYG{p}{,} \PYG{n}{o}\PYG{o}{.}\PYG{n}{sunrise}\PYG{p}{,} \PYG{l+s}{\PYGZsq{}}\PYG{l+s}{...}\PYG{l+s}{\PYGZsq{}}\PYG{p}{,} \PYG{n}{o}\PYG{o}{.}\PYG{n}{len\PYGZus{}night}\PYG{p}{)}
\PYG{g+go}{2015/3/31 18:08:40 ... 2015/4/1 05:13:09 ... 11.0746939826}
\PYG{g+gp}{\PYGZgt{}\PYGZgt{}\PYGZgt{} }\PYG{k+kn}{import} \PYG{n+nn}{ephem} \PYG{k+kn}{as} \PYG{n+nn}{E}
\PYG{g+gp}{\PYGZgt{}\PYGZgt{}\PYGZgt{} }\PYG{k}{print}\PYG{p}{(}\PYG{n}{E}\PYG{o}{.}\PYG{n}{Date}\PYG{p}{(}\PYG{n}{o}\PYG{o}{.}\PYG{n}{sunsetastro}\PYG{o}{+}\PYG{n}{o}\PYG{o}{.}\PYG{n}{localTimeOffest}\PYG{p}{)}\PYG{p}{,} \PYG{l+s}{\PYGZsq{}}\PYG{l+s}{...}\PYG{l+s}{\PYGZsq{}}\PYG{p}{,} \PYG{n}{E}\PYG{o}{.}\PYG{n}{Date}\PYG{p}{(}
\PYG{g+go}{        o.sunriseastro+o.localTimeOffest), \PYGZsq{}...\PYGZsq{}, o.len\PYGZus{}nightastro)}
\PYG{g+go}{2015/3/31 21:43:28 ... 2015/4/1 05:38:26 ... 7.91603336949}
\end{Verbatim}
\index{nowArg (astroobs.Observatory.Observatory attribute)}

\begin{fulllineitems}
\phantomsection\label{astroobs:astroobs.Observatory.Observatory.nowArg}\pysigline{\bfcode{nowArg}}
Returns the index of \emph{now} in the \code{observatory.dates} vector, or None if \emph{now} is out of its bounds (meaning the observation is not taking place now)

\begin{Verbatim}[commandchars=\\\{\}]
\PYG{g+gp}{\PYGZgt{}\PYGZgt{}\PYGZgt{} }\PYG{k+kn}{import} \PYG{n+nn}{astroobs.obs} \PYG{k+kn}{as} \PYG{n+nn}{obs}
\PYG{g+gp}{\PYGZgt{}\PYGZgt{}\PYGZgt{} }\PYG{k+kn}{import} \PYG{n+nn}{ephem} \PYG{k+kn}{as} \PYG{n+nn}{E}
\PYG{g+gp}{\PYGZgt{}\PYGZgt{}\PYGZgt{} }\PYG{n}{o} \PYG{o}{=} \PYG{n}{obs}\PYG{o}{.}\PYG{n}{Observatory}\PYG{p}{(}\PYG{l+s}{\PYGZsq{}}\PYG{l+s}{ohp}\PYG{l+s}{\PYGZsq{}}\PYG{p}{)}
\PYG{g+gp}{\PYGZgt{}\PYGZgt{}\PYGZgt{} }\PYG{n}{plt}\PYG{o}{.}\PYG{n}{plot}\PYG{p}{(}\PYG{n}{o}\PYG{o}{.}\PYG{n}{dates}\PYG{p}{,} \PYG{n}{o}\PYG{o}{.}\PYG{n}{moon}\PYG{o}{.}\PYG{n}{alt}\PYG{p}{,} \PYG{l+s}{\PYGZsq{}}\PYG{l+s}{k\PYGZhy{}}\PYG{l+s}{\PYGZsq{}}\PYG{p}{)}
\PYG{g+gp}{\PYGZgt{}\PYGZgt{}\PYGZgt{} }\PYG{n}{now} \PYG{o}{=} \PYG{n}{o}\PYG{o}{.}\PYG{n}{nowArg}
\PYG{g+gp}{\PYGZgt{}\PYGZgt{}\PYGZgt{} }\PYG{k}{if} \PYG{n}{now} \PYG{o+ow}{is} \PYG{o+ow}{not} \PYG{n+nb+bp}{None}\PYG{p}{:}
\PYG{g+gp}{\PYGZgt{}\PYGZgt{}\PYGZgt{} }    \PYG{n}{plt}\PYG{o}{.}\PYG{n}{plot}\PYG{p}{(}\PYG{n}{o}\PYG{o}{.}\PYG{n}{dates}\PYG{p}{[}\PYG{n}{now}\PYG{p}{]}\PYG{p}{,} \PYG{n}{o}\PYG{o}{.}\PYG{n}{moon}\PYG{o}{.}\PYG{n}{alt}\PYG{p}{[}\PYG{n}{now}\PYG{p}{]}\PYG{p}{,} \PYG{l+s}{\PYGZsq{}}\PYG{l+s}{ro}\PYG{l+s}{\PYGZsq{}}\PYG{p}{)}
\PYG{g+gp}{\PYGZgt{}\PYGZgt{}\PYGZgt{} }\PYG{k}{else}\PYG{p}{:}
\PYG{g+gp}{\PYGZgt{}\PYGZgt{}\PYGZgt{} }    \PYG{n}{plt}\PYG{o}{.}\PYG{n}{plot}\PYG{p}{(}\PYG{p}{[}\PYG{n}{E}\PYG{o}{.}\PYG{n}{now}\PYG{p}{(}\PYG{p}{)}\PYG{p}{,} \PYG{n}{E}\PYG{o}{.}\PYG{n}{now}\PYG{p}{(}\PYG{p}{)}\PYG{p}{]}\PYG{p}{,} \PYG{p}{[}\PYG{n}{o}\PYG{o}{.}\PYG{n}{moon}\PYG{o}{.}\PYG{n}{alt}\PYG{o}{.}\PYG{n}{min}\PYG{p}{(}\PYG{p}{)}\PYG{p}{,}\PYG{n}{o}\PYG{o}{.}\PYG{n}{moon}\PYG{o}{.}\PYG{n}{alt}\PYG{o}{.}\PYG{n}{max}\PYG{p}{(}\PYG{p}{)}\PYG{p}{]}\PYG{p}{,} \PYG{l+s}{\PYGZsq{}}\PYG{l+s}{r\PYGZhy{}\PYGZhy{}}\PYG{l+s}{\PYGZsq{}}\PYG{p}{)}
\end{Verbatim}

\end{fulllineitems}

\index{plot() (astroobs.Observatory.Observatory method)}

\begin{fulllineitems}
\phantomsection\label{astroobs:astroobs.Observatory.Observatory.plot}\pysiglinewithargsret{\bfcode{plot}}{\emph{**kwargs}}{}
Plots the observatory diagram
\begin{description}
\item[{Kwargs:}] \leavevmode\begin{itemize}
\item {} 
See class constructor

\item {} 
dt (float - hour): the spacing of x-axis labels, default is 1 hour (not with polar mode)

\item {} 
t0 (float - DJD or {[}0-24{]}): the date of the first tick-label of x-axis, default is sunsetastro. The time type must correspond to \code{time} parameter (not with polar mode)

\item {} 
xlim ({[}xmin, xmax{]}): bounds for x-axis, default is full night span (not with polar mode)

\item {} 
retxdisp (bool): if \code{True}, bounds of x-axis displayed values are returned (\code{xdisp} key)

\item {} 
ylim ({[}ymin, ymax{]}): bounds for y-axis, default is {[}horizon\_obs-10, 90{]} (not with polar mode)

\item {} 
xlabel (str): label for x-axis, default `Time (UT)'

\item {} 
ylabel (str): label for y-axis, default `Elevation (°)'

\item {} 
title (str): title of the diagram, default is observatory name or coordinates

\item {} 
ymin\_margin (float): margin between xmin of graph and horizon\_obs. Low priority vs ylim, default is 10 (not with polar mode)

\item {} 
retfignum (bool): if \code{True}, the figure number will be returned, default is \code{False}

\item {} 
fignum (int): figure number on which to plot, default is \code{False}

\item {} 
retaxnum (bool): if \code{True}, the ax index as in \code{figure.axes{[}n{]}} will be returned, default is \code{False}

\item {} 
axnum (int): axes index on which to plot, default is \code{None} (create new ax)

\item {} 
retfig (bool): if \code{True}, the figure object will be returned, default is \code{False}

\item {} 
fig (figure): figure object on which to plot, default is \code{None} (use fignum)

\item {} 
retax (bool): if \code{True}, the ax will be returned, default is \code{False}

\item {} 
ax (axes): ax on which to plot, default is \code{None}

\item {} 
now (bool): if \code{True} and within range, a vertical line as indication of ``now'' will be shown, default is True

\item {} 
retnow (bool): returns the line object (\code{nowline} key) corresponding to the `now-line', default is \code{False}

\item {} 
legend (bool): whether to add a legend or not, default is \code{True}

\item {} 
loc: location of the legend, default is 8 (top right), refer to plt.legend

\item {} 
ncol: number of columns in the legend, default is 3, refer to plt.legend

\item {} 
columnspacing: spacing between columns in the legend, refer to plt.legend

\item {} 
lfs: legend font size, default is 11

\item {} 
textlbl (bool): if \code{True}, a text label with target name or coordinates will be added near transit, default is \code{False}

\item {} 
polar (bool): if \code{True}, plots the sky view, otherwise plots target attribute versus time

\item {} 
time (str): the type of the x-axis time, \code{ut} for UT, \code{loc} for local time and \code{lst} {[}0-24{]} for local sidereal time, default is \code{ut} (not with polar mode)

\end{itemize}

\item[{Raises:}] \leavevmode
N/A

\end{description}

\end{fulllineitems}

\index{process\_obs() (astroobs.Observatory.Observatory method)}

\begin{fulllineitems}
\phantomsection\label{astroobs:astroobs.Observatory.Observatory.process_obs}\pysiglinewithargsret{\bfcode{process\_obs}}{\emph{pts=200}, \emph{margin=15}, \emph{fullhour=False}, \emph{**kwargs}}{}
Processes all twilights as well as moon rise, set and position through night for the given observatory and date.
Creates the vector \code{observatory.dates} which is the vector containing all timestamps at which the moon and the targets will be processed.
\begin{description}
\item[{Args:}] \leavevmode\begin{itemize}
\item {} 
pts (int) {[}optional{]}: the size of the \code{dates} vector, whose elements are linearly spaced in time

\item {} 
margin (float - minutes) {[}optional{]}: the margin between the first element of the vector \code{dates} and the sunset, and between the sunrise and its last element

\item {} 
fullhour (bool) {[}optional{]}: if \code{True}, then the vector \code{dates} will start and finish on the first full hour preceeding sunset and following sunrise

\end{itemize}

\item[{Kwargs:}] \leavevmode
See class constructor

\item[{Raises:}] \leavevmode\begin{itemize}
\item {} 
KeyError: if the twilight keyword is unknown

\item {} 
Exception: if the observatory object has no date

\end{itemize}

\end{description}

\begin{notice}{note}{Note:}
In case the observatory is in polar regions where the sun does not alway set and rise everyday, the first and last elements of the \code{dates} vector are set to local midday right before and after the local midnight of the observation date. e.g.: 24h night centered on the local midnight.
\end{notice}

\end{fulllineitems}

\index{upd\_date() (astroobs.Observatory.Observatory method)}

\begin{fulllineitems}
\phantomsection\label{astroobs:astroobs.Observatory.Observatory.upd_date}\pysiglinewithargsret{\bfcode{upd\_date}}{\emph{ut\_date=None}, \emph{local\_date=None}, \emph{force=False}, \emph{**kwargs}}{}
Updates the date of the observatory, and re-process the observatory parameters if the date is different.
\begin{description}
\item[{Args:}] \leavevmode\begin{itemize}
\item {} 
ut\_date (see below): the date of observation in UT time

\item {} 
local\_date (see below): the date of observation in local time

\item {} 
force (bool): if \code{False}, the observatory is re-processed only if the date changed

\end{itemize}

\item[{Kwargs:}] \leavevmode
See class constructor

\item[{Raises:}] \leavevmode\begin{itemize}
\item {} 
KeyError: if the twilight keyword is unknown

\item {} 
Exception: if the observatory object has no date

\end{itemize}

\item[{Returns:}] \leavevmode
\code{True} if the date was changed, otherwise \code{False}

\end{description}

\begin{notice}{note}{Note:}\begin{itemize}
\item {} 
\code{local\_date} and \code{ut\_date} can be date-tuples \code{(yyyy, mm, dd, {[}hh, mm, ss{]})}, timestamps, datetime structures or ephem.Date instances.

\item {} 
If both are given, \code{ut\_date} has higher priority

\item {} 
If neither of those are given, the date is automatically set to \emph{tonight} or \emph{now} (whether the sun has already set or not)

\end{itemize}
\end{notice}

\end{fulllineitems}


\end{fulllineitems}



\section{astroobs.ObservatoryList module}
\label{astroobs:astroobs-observatorylist-module}\label{astroobs:module-astroobs.ObservatoryList}\index{astroobs.ObservatoryList (module)}\index{ObservatoryList (class in astroobs.ObservatoryList)}

\begin{fulllineitems}
\phantomsection\label{astroobs:astroobs.ObservatoryList.ObservatoryList}\pysiglinewithargsret{\strong{class }\code{astroobs.ObservatoryList.}\bfcode{ObservatoryList}}{\emph{dataFile=None}, \emph{**kwargs}}{}
Bases: \code{object}

Manages the database of observatories.
\begin{description}
\item[{Args:}] \leavevmode\begin{itemize}
\item {} 
dataFile (str): path+file to the observatories database. If left to \code{None}, the standard package database will be used

\end{itemize}

\item[{Kwargs:}] \leavevmode\begin{itemize}
\item {} 
raiseError (bool): if \code{True}, errors will be raised; if \code{False}, they will be printed. Default is \code{False}

\end{itemize}

\item[{Raises:}] \leavevmode\begin{itemize}
\item {} 
Exception: if a mandatory input parameter is missing when loading all observatories

\end{itemize}

\end{description}

Use \code{add()}, \code{rem()}, \code{mod()} to add, remove or modify an observatory to the database.

\begin{Verbatim}[commandchars=\\\{\}]
\PYG{g+gp}{\PYGZgt{}\PYGZgt{}\PYGZgt{} }\PYG{k+kn}{import} \PYG{n+nn}{astroobs.obs} \PYG{k+kn}{as} \PYG{n+nn}{obs}
\PYG{g+gp}{\PYGZgt{}\PYGZgt{}\PYGZgt{} }\PYG{n}{ol} \PYG{o}{=} \PYG{n}{obs}\PYG{o}{.}\PYG{n}{ObservatoryList}\PYG{p}{(}\PYG{p}{)}
\PYG{g+gp}{\PYGZgt{}\PYGZgt{}\PYGZgt{} }\PYG{n}{ol}
\PYG{g+go}{List of 34 observatories}
\PYG{g+gp}{\PYGZgt{}\PYGZgt{}\PYGZgt{} }\PYG{n}{ol}\PYG{o}{.}\PYG{n}{obsids}
\PYG{g+go}{[\PYGZsq{}mwo\PYGZsq{},}
\PYG{g+go}{ \PYGZsq{}kpno\PYGZsq{},}
\PYG{g+go}{ \PYGZsq{}ctio\PYGZsq{},}
\PYG{g+go}{ \PYGZsq{}lasilla\PYGZsq{},}
\PYG{g+go}{ ...}
\PYG{g+go}{ \PYGZsq{}vlt\PYGZsq{},}
\PYG{g+go}{ \PYGZsq{}mgo\PYGZsq{},}
\PYG{g+go}{ \PYGZsq{}ohp\PYGZsq{}]}
\PYG{g+gp}{\PYGZgt{}\PYGZgt{}\PYGZgt{} }\PYG{n}{ol}\PYG{p}{[}\PYG{l+s}{\PYGZsq{}}\PYG{l+s}{ohp}\PYG{l+s}{\PYGZsq{}}\PYG{p}{]}
\PYG{g+go}{\PYGZob{}\PYGZsq{}elevation\PYGZsq{}: 650.0,}
\PYG{g+go}{ \PYGZsq{}lat\PYGZsq{}: 0.7667376848115423,}
\PYG{g+go}{ \PYGZsq{}long\PYGZsq{}: 0.09971647793060935,}
\PYG{g+go}{ \PYGZsq{}moonAvoidRadius\PYGZsq{}: 0.25,}
\PYG{g+go}{ \PYGZsq{}name\PYGZsq{}: \PYGZsq{}Observatoire de Haute Provence\PYGZsq{},}
\PYG{g+go}{ \PYGZsq{}pressure\PYGZsq{}: 1010.0,}
\PYG{g+go}{ \PYGZsq{}temp\PYGZsq{}: 15.0,}
\PYG{g+go}{ \PYGZsq{}timezone\PYGZsq{}: \PYGZsq{}Europe/Paris\PYGZsq{}\PYGZcb{}}
\end{Verbatim}
\index{add() (astroobs.ObservatoryList.ObservatoryList method)}

\begin{fulllineitems}
\phantomsection\label{astroobs:astroobs.ObservatoryList.ObservatoryList.add}\pysiglinewithargsret{\bfcode{add}}{\emph{obsid}, \emph{name}, \emph{long}, \emph{lat}, \emph{elevation}, \emph{timezone}, \emph{temp=15.0}, \emph{pressure=1010.0}, \emph{moonAvoidRadius=0.25}, \emph{**kwargs}}{}
Adds an observatory to the current observatories database.
\begin{description}
\item[{Args:}] \leavevmode\begin{itemize}
\item {} 
obsid (str): id of the observatory to add. Must be unique, without spaces or ;

\item {} 
name (str): name of the observatory

\item {} 
long (str - `+/-ddd:mm:ss.s'): longitude of the observatory. West is negative, East is positive

\item {} 
lat (str - `+/-dd:mm:ss.s'): latitude of the observatory. North is Positive, South is negative

\item {} 
elevation (float - m): elevation of the observatory

\item {} 
timezone (str): timezone of the observatory, as in pytz library. See note below

\item {} 
temp (float - degrees Celcius) {[}optional{]}: temperature at the observatory

\item {} 
pressure (float - hPa) {[}optional{]}: pressure at the observatory

\item {} 
moonAvoidRadius (float - degrees) {[}optional{]}: minimum distance at which a target must sit from the moon to be observed

\end{itemize}

\item[{Kwargs:}] \leavevmode
See class constructor

\item[{Raises:}] \leavevmode\begin{itemize}
\item {} 
NameError: if the observatory ID already exists

\item {} 
Exception: if a mandatory input parameter is missing when reloading all observatories

\end{itemize}

\end{description}

\begin{notice}{note}{Note:}
To view all available timezones, run:
\textgreater{}\textgreater{}\textgreater{} import pytz
\textgreater{}\textgreater{}\textgreater{} for tz in pytz.all\_timezones:
\textgreater{}\textgreater{}\textgreater{}     print(tz)
\end{notice}

\end{fulllineitems}

\index{mod() (astroobs.ObservatoryList.ObservatoryList method)}

\begin{fulllineitems}
\phantomsection\label{astroobs:astroobs.ObservatoryList.ObservatoryList.mod}\pysiglinewithargsret{\bfcode{mod}}{\emph{obsid}, \emph{name}, \emph{long}, \emph{lat}, \emph{elevation}, \emph{timezone}, \emph{temp=15.0}, \emph{pressure=1010.0}, \emph{moonAvoidRadius=0.25}, \emph{**kwargs}}{}
Modifies an observatory in the current observatories database.
\begin{description}
\item[{Args:}] \leavevmode\begin{itemize}
\item {} 
obsid (str): id of the observatory to modify. All other parameters redefine the observatory

\end{itemize}

\item[{Kwargs:}] \leavevmode
See class constructor

\item[{Raises:}] \leavevmode\begin{itemize}
\item {} 
NameError: if the observatory ID does not exist

\item {} 
Exception: if a mandatory input parameter is missing when reloading all observatories

\end{itemize}

\end{description}

\begin{notice}{note}{Note:}
Refer to \code{add()} for details on input parameters
\end{notice}

\end{fulllineitems}

\index{nameList() (astroobs.ObservatoryList.ObservatoryList method)}

\begin{fulllineitems}
\phantomsection\label{astroobs:astroobs.ObservatoryList.ObservatoryList.nameList}\pysiglinewithargsret{\bfcode{nameList}}{}{}
Provides a list of tuples (obs id, observatory name) in the alphabetical order of the column `observatory name'.

\end{fulllineitems}

\index{rem() (astroobs.ObservatoryList.ObservatoryList method)}

\begin{fulllineitems}
\phantomsection\label{astroobs:astroobs.ObservatoryList.ObservatoryList.rem}\pysiglinewithargsret{\bfcode{rem}}{\emph{obsid}, \emph{**kwargs}}{}
Removes an observatory from the current observatories database.
\begin{description}
\item[{Args:}] \leavevmode\begin{itemize}
\item {} 
obsid (str): id of the observatory to remove

\end{itemize}

\item[{Kwargs:}] \leavevmode
See class constructor

\item[{Raises:}] \leavevmode\begin{itemize}
\item {} 
NameError: if the observatory ID does not exist

\item {} 
Exception: if a mandatory input parameter is missing when reloading all observatories

\end{itemize}

\end{description}

\end{fulllineitems}


\end{fulllineitems}

\index{showall() (in module astroobs.ObservatoryList)}

\begin{fulllineitems}
\phantomsection\label{astroobs:astroobs.ObservatoryList.showall}\pysiglinewithargsret{\code{astroobs.ObservatoryList.}\bfcode{showall}}{\emph{dataFile=None}, \emph{**kwargs}}{}
A quick function to view all available observatories

\end{fulllineitems}



\section{astroobs.Target module}
\label{astroobs:astroobs-target-module}\label{astroobs:module-astroobs.Target}\index{astroobs.Target (module)}\index{Target (class in astroobs.Target)}

\begin{fulllineitems}
\phantomsection\label{astroobs:astroobs.Target.Target}\pysiglinewithargsret{\strong{class }\code{astroobs.Target.}\bfcode{Target}}{\emph{ra}, \emph{dec}, \emph{name}, \emph{input\_epoch=`2000'}, \emph{obs=None}, \emph{**kwargs}}{}
Bases: \code{object}

Initialises a target object from its right ascension and declination. Optionaly, processes the target for the observatory and date given (refer to \code{Target.process()}).
\begin{description}
\item[{Args:}] \leavevmode\begin{itemize}
\item {} 
ra (str `hh:mm:ss.s' or float - degrees): the right ascension of the target

\item {} 
dec (str `+/-dd:mm:ss.s' or float - degrees): the declination of the target

\item {} 
name (str): the name of the target, for display

\item {} 
obs (\code{Observatory}) {[}optional{]}: the observatory for which to process the target

\item {} 
input\_epoch (str): the `YYYY' year of epoch in which the ra-dec coordinates are given. These coordinates will corrected with precession if the epoch of observatory is different

\end{itemize}

\item[{Kwargs:}] \leavevmode\begin{itemize}
\item {} 
raiseError (bool): if \code{True}, errors will be raised; if \code{False}, they will be printed. Default is \code{False}

\end{itemize}

\item[{Raises:}] \leavevmode
N/A

\end{description}
\index{dec (astroobs.Target.Target attribute)}

\begin{fulllineitems}
\phantomsection\label{astroobs:astroobs.Target.Target.dec}\pysigline{\bfcode{dec}}
The declination of the target, displayed as tuple (+/-dd, mm, ss)

\end{fulllineitems}

\index{decStr (astroobs.Target.Target attribute)}

\begin{fulllineitems}
\phantomsection\label{astroobs:astroobs.Target.Target.decStr}\pysigline{\bfcode{decStr}}
A pretty printable version of the declination of the target

\end{fulllineitems}

\index{plot() (astroobs.Target.Target method)}

\begin{fulllineitems}
\phantomsection\label{astroobs:astroobs.Target.Target.plot}\pysiglinewithargsret{\bfcode{plot}}{\emph{obs}, \emph{y='alt'}, \emph{**kwargs}}{}
Plots the y-parameter vs time diagram for the target at the given observatory and date
\begin{description}
\item[{Args:}] \leavevmode\begin{itemize}
\item {} 
obs (\code{Observatory}): the observatory for which to plot the target

\item {} 
y (object attribute): the y-data to plot

\end{itemize}

\item[{Kwargs:}] \leavevmode\begin{itemize}
\item {} 
See class constructor

\item {} 
See \code{Observatory.plot()}

\item {} 
simpleplt (bool): if \code{True}, the observatory plot will not be plotted, default is \code{False}

\item {} 
color (str or \#XXXXXX): the color of the target curve, default is `k'

\item {} 
lw (float): the linewidth, default is 1

\end{itemize}

\item[{Raises:}] \leavevmode
N/A

\end{description}

\end{fulllineitems}

\index{polar() (astroobs.Target.Target method)}

\begin{fulllineitems}
\phantomsection\label{astroobs:astroobs.Target.Target.polar}\pysiglinewithargsret{\bfcode{polar}}{\emph{obs}, \emph{**kwargs}}{}
Plots the sky-view diagram for the target at the given observatory and date
\begin{description}
\item[{Args:}] \leavevmode\begin{itemize}
\item {} 
obs (\code{Observatory}): the observatory for which to plot the target

\item {} 
y (object attribute): the y-data to plot

\end{itemize}

\item[{Kwargs:}] \leavevmode\begin{itemize}
\item {} 
See class constructor

\item {} 
See \code{Observatory.plot()}

\item {} 
See \code{Target.plot()}

\end{itemize}

\item[{Raises:}] \leavevmode
N/A

\end{description}

\end{fulllineitems}

\index{process() (astroobs.Target.Target method)}

\begin{fulllineitems}
\phantomsection\label{astroobs:astroobs.Target.Target.process}\pysiglinewithargsret{\bfcode{process}}{\emph{obs}, \emph{**kwargs}}{}
Processes the target for the given observatory and date.
\begin{description}
\item[{Args:}] \leavevmode\begin{itemize}
\item {} 
obs (\code{Observatory}): the observatory for which to process the target

\end{itemize}

\item[{Kwargs:}] \leavevmode
See class constructor

\item[{Raises:}] \leavevmode
N/A

\item[{Creates vector attributes:}] \leavevmode\begin{itemize}
\item {} 
\code{airmass}: the airmass of the target

\item {} 
\code{ha}: the hour angle of the target (degrees)

\item {} 
\code{alt}: the altitude of the target (degrees - horizon is 0)

\item {} 
\code{az}: the azimuth of the target (degrees)

\item {} 
\code{moondist}: the angular distance between the moon and the target (degrees)

\end{itemize}

\end{description}

\begin{notice}{note}{Note:}\begin{itemize}
\item {} 
All previous attributes are vectors related to the time vector of the observatory used for processing, stored under \code{dates} attribute

\end{itemize}
\end{notice}
\begin{description}
\item[{Other attributes:}] \leavevmode\begin{itemize}
\item {} 
\code{rise\_time}, \code{rise\_az}: the time (ephem.Date) and the azimuth (degree) of the rise of the target

\item {} 
\code{set\_time}, \code{set\_az}: the time (ephem.Date) and the azimuth (degree) of the setting of the target

\item {} 
\code{transit\_time}, \code{transit\_az}: the time (ephem.Date) and the azimuth (degree) of the transit of the target

\end{itemize}

\end{description}

\begin{notice}{warning}{Warning:}\begin{itemize}
\item {} 
it can occur that the target does not rise or set for an observatory/date combination. In that case, the corresponding attributes will be set to \code{None}, i.e. \code{set\_time}, \code{set\_az}, \code{rise\_time}, \code{rise\_az}. In that case, an additional parameter is added to the Target object: \code{Target.alwaysUp} which is \code{True} if the target never sets and \code{False} if it never rises above the horizon.

\end{itemize}
\end{notice}

\end{fulllineitems}

\index{ra (astroobs.Target.Target attribute)}

\begin{fulllineitems}
\phantomsection\label{astroobs:astroobs.Target.Target.ra}\pysigline{\bfcode{ra}}
The right ascension of the target, displayed as tuple (hh, mm, ss)

\end{fulllineitems}

\index{raStr (astroobs.Target.Target attribute)}

\begin{fulllineitems}
\phantomsection\label{astroobs:astroobs.Target.Target.raStr}\pysigline{\bfcode{raStr}}
A pretty printable version of the right ascension of the target

\end{fulllineitems}

\index{whenobs() (astroobs.Target.Target method)}

\begin{fulllineitems}
\phantomsection\label{astroobs:astroobs.Target.Target.whenobs}\pysiglinewithargsret{\bfcode{whenobs}}{\emph{obs}, \emph{fromDate='now'}, \emph{toDate='now+30day'}, \emph{plot=True}, \emph{ret=False}, \emph{dday=1}, \emph{**kwargs}}{}
Processes the target for the given observatory and dat.
\begin{description}
\item[{Args:}] \leavevmode\begin{itemize}
\item {} 
obs (\code{Observatory}): the observatory for which to process the target

\item {} 
fromDate (see below): the start date of the range

\item {} 
toDate (see below): the end date of the range

\item {} 
plot: whether it plots the diagram

\item {} 
ret: whether it returns the values

\item {} 
dday: the

\end{itemize}

\item[{Kwargs:}] \leavevmode
See class constructor
* legend (bool): whether to add a legend or not, default is \code{True}
* loc: location of the legend, default is 8 (top right), refer to plt.legend
* ncol: number of columns in the legend, default is 3, refer to plt.legend
* columnspacing: spacing between columns in the legend, refer to plt.legend
* lfs: legend font size, default is 11

\item[{Raises:}] \leavevmode
N/A

\end{description}

\begin{notice}{note}{Note:}\begin{itemize}
\item {} 
\code{local\_date} and \code{ut\_date} can be date-tuples \code{(yyyy, mm, dd, {[}hh, mm, ss{]})}, timestamps, datetime structures or ephem.Date instances.

\end{itemize}
\end{notice}

\end{fulllineitems}


\end{fulllineitems}



\section{astroobs.TargetSIMBAD module}
\label{astroobs:astroobs-targetsimbad-module}\label{astroobs:module-astroobs.TargetSIMBAD}\index{astroobs.TargetSIMBAD (module)}\index{TargetSIMBAD (class in astroobs.TargetSIMBAD)}

\begin{fulllineitems}
\phantomsection\label{astroobs:astroobs.TargetSIMBAD.TargetSIMBAD}\pysiglinewithargsret{\strong{class }\code{astroobs.TargetSIMBAD.}\bfcode{TargetSIMBAD}}{\emph{name}, \emph{obs=None}, \emph{input\_epoch=`2000'}, \emph{**kwargs}}{}
Bases: \code{astroobs.Target.Target}

Initialises a target object from an online SIMBAD database name-search. Optionaly, processes the target for the observatory and date given (refer to \code{TargetSIMBAD.process()}).
\begin{description}
\item[{Args:}] \leavevmode\begin{itemize}
\item {} 
name (str): the name of the target as if performing an online SIMBAD search

\item {} 
obs (\code{Observatory}) {[}optional{]}: the observatory for which to process the target

\end{itemize}

\item[{Kwargs:}] \leavevmode\begin{itemize}
\item {} 
raiseError (bool): if \code{True}, errors will be raised; if \code{False}, they will be printed. Default is \code{False}

\end{itemize}

\item[{Raises:}] \leavevmode
N/A

\item[{Creates attributes:}] \leavevmode\begin{itemize}
\item {} 
\code{flux}: a dictionary of the magnitudes of the target. Keys are part or all of {[}'U','B','V','R','I','J','H','K'{]}

\item {} 
\code{link}: the link to paste into a web-browser to display the SIMBAD page of the target

\item {} 
\code{linkbib}: the link to paste into a web-browser to display the references on the SIMBAD page of the target

\item {} 
\code{hd}: if applicable, the HD number of the target

\item {} 
\code{hr}: if applicable, the HR number of the target

\item {} 
\code{hip}: if applicable, the HIP number of the target

\end{itemize}

\end{description}

\end{fulllineitems}



\section{astroobs.astroobsexception module}
\label{astroobs:astroobs-astroobsexception-module}\label{astroobs:module-astroobs.astroobsexception}\index{astroobs.astroobsexception (module)}\index{AstroobsException}

\begin{fulllineitems}
\phantomsection\label{astroobs:astroobs.astroobsexception.AstroobsException}\pysigline{\strong{exception }\code{astroobs.astroobsexception.}\bfcode{AstroobsException}}
Bases: \code{exceptions.Exception}

Root for astroobs Exceptions, only used to except any astroobs error, never raised

\end{fulllineitems}

\index{DuplicateObservatory}

\begin{fulllineitems}
\phantomsection\label{astroobs:astroobs.astroobsexception.DuplicateObservatory}\pysiglinewithargsret{\strong{exception }\code{astroobs.astroobsexception.}\bfcode{DuplicateObservatory}}{\emph{key='`}, \emph{*args}}{}
Bases: \code{astroobs.astroobsexception.AstroobsException}

If the observatory key is already existing

\end{fulllineitems}

\index{InputNotUnderstood}

\begin{fulllineitems}
\phantomsection\label{astroobs:astroobs.astroobsexception.InputNotUnderstood}\pysiglinewithargsret{\strong{exception }\code{astroobs.astroobsexception.}\bfcode{InputNotUnderstood}}{\emph{ipt='`}, \emph{*args}}{}
Bases: \code{astroobs.astroobsexception.AstroobsException}

If the input was not understood

\end{fulllineitems}

\index{NoObservatoryDate}

\begin{fulllineitems}
\phantomsection\label{astroobs:astroobs.astroobsexception.NoObservatoryDate}\pysiglinewithargsret{\strong{exception }\code{astroobs.astroobsexception.}\bfcode{NoObservatoryDate}}{\emph{*args}}{}
Bases: \code{astroobs.astroobsexception.UncompleteObservatory}

If the observatory is missing a date

\end{fulllineitems}

\index{NoPlotMode}

\begin{fulllineitems}
\phantomsection\label{astroobs:astroobs.astroobsexception.NoPlotMode}\pysiglinewithargsret{\strong{exception }\code{astroobs.astroobsexception.}\bfcode{NoPlotMode}}{\emph{*args}}{}
Bases: \code{astroobs.astroobsexception.AstroobsException}

If the user doesn't have matplotlib

\end{fulllineitems}

\index{NonObservatory}

\begin{fulllineitems}
\phantomsection\label{astroobs:astroobs.astroobsexception.NonObservatory}\pysiglinewithargsret{\strong{exception }\code{astroobs.astroobsexception.}\bfcode{NonObservatory}}{\emph{obs='`}, \emph{*args}}{}
Bases: \code{astroobs.astroobsexception.AstroobsException}

If one or more parameter is missing in the setting up of the Obervatory object

\end{fulllineitems}

\index{NonObservatoryList}

\begin{fulllineitems}
\phantomsection\label{astroobs:astroobs.astroobsexception.NonObservatoryList}\pysiglinewithargsret{\strong{exception }\code{astroobs.astroobsexception.}\bfcode{NonObservatoryList}}{\emph{*args}}{}
Bases: \code{astroobs.astroobsexception.AstroobsException}

If the observatory list is not valid

\end{fulllineitems}

\index{NonTarget}

\begin{fulllineitems}
\phantomsection\label{astroobs:astroobs.astroobsexception.NonTarget}\pysiglinewithargsret{\strong{exception }\code{astroobs.astroobsexception.}\bfcode{NonTarget}}{\emph{obj='`}, \emph{*args}}{}
Bases: \code{astroobs.astroobsexception.AstroobsException}

If the type of the object is not astroobs.Target, or is not valid

\end{fulllineitems}

\index{ReadOnly}

\begin{fulllineitems}
\phantomsection\label{astroobs:astroobs.astroobsexception.ReadOnly}\pysiglinewithargsret{\strong{exception }\code{astroobs.astroobsexception.}\bfcode{ReadOnly}}{\emph{attr='`}, \emph{*args}}{}
Bases: \code{astroobs.astroobsexception.AstroobsException}

If the parameter is read-only

\end{fulllineitems}

\index{TargetMissingSIMBAD}

\begin{fulllineitems}
\phantomsection\label{astroobs:astroobs.astroobsexception.TargetMissingSIMBAD}\pysiglinewithargsret{\strong{exception }\code{astroobs.astroobsexception.}\bfcode{TargetMissingSIMBAD}}{\emph{target='`}, \emph{*args}}{}
Bases: \code{astroobs.astroobsexception.AstroobsException}

If the target name given was not found in SIMBAD

\end{fulllineitems}

\index{UncompleteObservatory}

\begin{fulllineitems}
\phantomsection\label{astroobs:astroobs.astroobsexception.UncompleteObservatory}\pysiglinewithargsret{\strong{exception }\code{astroobs.astroobsexception.}\bfcode{UncompleteObservatory}}{\emph{param='`}, \emph{*args}}{}
Bases: \code{astroobs.astroobsexception.AstroobsException}

If one or more parameter is missing in the setting up of the Obervatory object

\end{fulllineitems}

\index{UnknownObservatory}

\begin{fulllineitems}
\phantomsection\label{astroobs:astroobs.astroobsexception.UnknownObservatory}\pysiglinewithargsret{\strong{exception }\code{astroobs.astroobsexception.}\bfcode{UnknownObservatory}}{\emph{obs='`}, \emph{*args}}{}
Bases: \code{astroobs.astroobsexception.AstroobsException}

If the observatory key is not known

\end{fulllineitems}

\index{UnknownTwilight}

\begin{fulllineitems}
\phantomsection\label{astroobs:astroobs.astroobsexception.UnknownTwilight}\pysiglinewithargsret{\strong{exception }\code{astroobs.astroobsexception.}\bfcode{UnknownTwilight}}{\emph{twi='`}, \emph{*args}}{}
Bases: \code{astroobs.astroobsexception.AstroobsException}

If the twilight key is not known

\end{fulllineitems}

\index{raiseIt() (in module astroobs.astroobsexception)}

\begin{fulllineitems}
\phantomsection\label{astroobs:astroobs.astroobsexception.raiseIt}\pysiglinewithargsret{\code{astroobs.astroobsexception.}\bfcode{raiseIt}}{\emph{exc}, \emph{raiseoupas}, \emph{*args}}{}
\end{fulllineitems}



\section{astroobs.core module}
\label{astroobs:astroobs-core-module}\label{astroobs:module-astroobs.core}\index{astroobs.core (module)}\index{airmass\_to\_rad() (in module astroobs.core)}

\begin{fulllineitems}
\phantomsection\label{astroobs:astroobs.core.airmass_to_rad}\pysiglinewithargsret{\code{astroobs.core.}\bfcode{airmass\_to\_rad}}{\emph{arr}}{}
Transforms airmass to radians

\end{fulllineitems}

\index{cleanTime() (in module astroobs.core)}

\begin{fulllineitems}
\phantomsection\label{astroobs:astroobs.core.cleanTime}\pysiglinewithargsret{\code{astroobs.core.}\bfcode{cleanTime}}{\emph{t}, \emph{format=None}}{}
Raises an error if t not among (ephem.Date, datetime, timestamp, tuple, time.struct\_time) date types, and optionaly returns the date into the format:
- `ts': unix timestamp (float)
- `dt': datetime
- `du': date tuple
- `ed': ephem.Date
- `st': time.struct\_time

NB: does not keep the tzinfo of datetime

\end{fulllineitems}

\index{convertTime() (in module astroobs.core)}

\begin{fulllineitems}
\phantomsection\label{astroobs:astroobs.core.convertTime}\pysiglinewithargsret{\code{astroobs.core.}\bfcode{convertTime}}{\emph{t}, \emph{tzTo}, \emph{tzFrom='utc'}, \emph{format=None}}{}
Converts the time `t' from timezone `tzFrom' (default is UT) to timezone `tzTo'.

tzFrom and tzTo are like `America/Los\_Angeles'

cf cleanTime method to see possible types for `t' and output.

\end{fulllineitems}

\index{make\_num() (in module astroobs.core)}

\begin{fulllineitems}
\phantomsection\label{astroobs:astroobs.core.make_num}\pysiglinewithargsret{\code{astroobs.core.}\bfcode{make\_num}}{\emph{numstr}}{}
Removes any non-number character from numstr. Keeps also decimal separator ''.'' and signs ``-'', ``+''.
Returns float

\end{fulllineitems}

\index{rad\_to\_airmass() (in module astroobs.core)}

\begin{fulllineitems}
\phantomsection\label{astroobs:astroobs.core.rad_to_airmass}\pysiglinewithargsret{\code{astroobs.core.}\bfcode{rad\_to\_airmass}}{\emph{arr}}{}
Transforms radians to airmass

\end{fulllineitems}

\index{radecFromStr() (in module astroobs.core)}

\begin{fulllineitems}
\phantomsection\label{astroobs:astroobs.core.radecFromStr}\pysiglinewithargsret{\code{astroobs.core.}\bfcode{radecFromStr}}{\emph{txt}}{}
Takes a string that contains ra in decimal degrees or in hh:mm:ss.s and dec in decimal degrees or dd:mm:ss.s
returns (ra, dec) in decimal degrees

\end{fulllineitems}



\section{astroobs.obs module}
\label{astroobs:astroobs-obs-module}\label{astroobs:module-astroobs.obs}\index{astroobs.obs (module)}

\section{astroobs.version module}
\label{astroobs:astroobs-version-module}\label{astroobs:module-astroobs.version}\index{astroobs.version (module)}

\section{Module contents}
\label{astroobs:module-contents}\label{astroobs:module-astroobs}\index{astroobs (module)}
Provides astronomy ephemeris to plan telescope observations

\begin{notice}{note}{Note:}\begin{itemize}
\item {} 
All altitudes, azimuth, hour angle are in degrees

\item {} 
\code{horizon} attribute of \code{Observatory} or \code{Observation} is in radian

\item {} 
All times are in UT, except for \code{Observatory.localnight}

\end{itemize}
\end{notice}

\begin{notice}{warning}{Warning:}\begin{itemize}
\item {} 
it can occur that the Sun, the Moon or a target does not rise or set for an observatory/date combination. In that case, the corresponding attributes will be set to \code{None}

\end{itemize}
\end{notice}

Real-life example use:
\textgreater{}\textgreater{}\textgreater{}
\index{ObservatoryList (class in astroobs)}

\begin{fulllineitems}
\phantomsection\label{astroobs:astroobs.ObservatoryList}\pysiglinewithargsret{\strong{class }\code{astroobs.}\bfcode{ObservatoryList}}{\emph{dataFile=None}, \emph{**kwargs}}{}
Bases: \code{object}

Manages the database of observatories.
\begin{description}
\item[{Args:}] \leavevmode\begin{itemize}
\item {} 
dataFile (str): path+file to the observatories database. If left to \code{None}, the standard package database will be used

\end{itemize}

\item[{Kwargs:}] \leavevmode\begin{itemize}
\item {} 
raiseError (bool): if \code{True}, errors will be raised; if \code{False}, they will be printed. Default is \code{False}

\end{itemize}

\item[{Raises:}] \leavevmode\begin{itemize}
\item {} 
Exception: if a mandatory input parameter is missing when loading all observatories

\end{itemize}

\end{description}

Use \code{add()}, \code{rem()}, \code{mod()} to add, remove or modify an observatory to the database.

\begin{Verbatim}[commandchars=\\\{\}]
\PYG{g+gp}{\PYGZgt{}\PYGZgt{}\PYGZgt{} }\PYG{k+kn}{import} \PYG{n+nn}{astroobs.obs} \PYG{k+kn}{as} \PYG{n+nn}{obs}
\PYG{g+gp}{\PYGZgt{}\PYGZgt{}\PYGZgt{} }\PYG{n}{ol} \PYG{o}{=} \PYG{n}{obs}\PYG{o}{.}\PYG{n}{ObservatoryList}\PYG{p}{(}\PYG{p}{)}
\PYG{g+gp}{\PYGZgt{}\PYGZgt{}\PYGZgt{} }\PYG{n}{ol}
\PYG{g+go}{List of 34 observatories}
\PYG{g+gp}{\PYGZgt{}\PYGZgt{}\PYGZgt{} }\PYG{n}{ol}\PYG{o}{.}\PYG{n}{obsids}
\PYG{g+go}{[\PYGZsq{}mwo\PYGZsq{},}
\PYG{g+go}{ \PYGZsq{}kpno\PYGZsq{},}
\PYG{g+go}{ \PYGZsq{}ctio\PYGZsq{},}
\PYG{g+go}{ \PYGZsq{}lasilla\PYGZsq{},}
\PYG{g+go}{ ...}
\PYG{g+go}{ \PYGZsq{}vlt\PYGZsq{},}
\PYG{g+go}{ \PYGZsq{}mgo\PYGZsq{},}
\PYG{g+go}{ \PYGZsq{}ohp\PYGZsq{}]}
\PYG{g+gp}{\PYGZgt{}\PYGZgt{}\PYGZgt{} }\PYG{n}{ol}\PYG{p}{[}\PYG{l+s}{\PYGZsq{}}\PYG{l+s}{ohp}\PYG{l+s}{\PYGZsq{}}\PYG{p}{]}
\PYG{g+go}{\PYGZob{}\PYGZsq{}elevation\PYGZsq{}: 650.0,}
\PYG{g+go}{ \PYGZsq{}lat\PYGZsq{}: 0.7667376848115423,}
\PYG{g+go}{ \PYGZsq{}long\PYGZsq{}: 0.09971647793060935,}
\PYG{g+go}{ \PYGZsq{}moonAvoidRadius\PYGZsq{}: 0.25,}
\PYG{g+go}{ \PYGZsq{}name\PYGZsq{}: \PYGZsq{}Observatoire de Haute Provence\PYGZsq{},}
\PYG{g+go}{ \PYGZsq{}pressure\PYGZsq{}: 1010.0,}
\PYG{g+go}{ \PYGZsq{}temp\PYGZsq{}: 15.0,}
\PYG{g+go}{ \PYGZsq{}timezone\PYGZsq{}: \PYGZsq{}Europe/Paris\PYGZsq{}\PYGZcb{}}
\end{Verbatim}
\index{add() (astroobs.ObservatoryList method)}

\begin{fulllineitems}
\phantomsection\label{astroobs:astroobs.ObservatoryList.add}\pysiglinewithargsret{\bfcode{add}}{\emph{obsid}, \emph{name}, \emph{long}, \emph{lat}, \emph{elevation}, \emph{timezone}, \emph{temp=15.0}, \emph{pressure=1010.0}, \emph{moonAvoidRadius=0.25}, \emph{**kwargs}}{}
Adds an observatory to the current observatories database.
\begin{description}
\item[{Args:}] \leavevmode\begin{itemize}
\item {} 
obsid (str): id of the observatory to add. Must be unique, without spaces or ;

\item {} 
name (str): name of the observatory

\item {} 
long (str - `+/-ddd:mm:ss.s'): longitude of the observatory. West is negative, East is positive

\item {} 
lat (str - `+/-dd:mm:ss.s'): latitude of the observatory. North is Positive, South is negative

\item {} 
elevation (float - m): elevation of the observatory

\item {} 
timezone (str): timezone of the observatory, as in pytz library. See note below

\item {} 
temp (float - degrees Celcius) {[}optional{]}: temperature at the observatory

\item {} 
pressure (float - hPa) {[}optional{]}: pressure at the observatory

\item {} 
moonAvoidRadius (float - degrees) {[}optional{]}: minimum distance at which a target must sit from the moon to be observed

\end{itemize}

\item[{Kwargs:}] \leavevmode
See class constructor

\item[{Raises:}] \leavevmode\begin{itemize}
\item {} 
NameError: if the observatory ID already exists

\item {} 
Exception: if a mandatory input parameter is missing when reloading all observatories

\end{itemize}

\end{description}

\begin{notice}{note}{Note:}
To view all available timezones, run:
\textgreater{}\textgreater{}\textgreater{} import pytz
\textgreater{}\textgreater{}\textgreater{} for tz in pytz.all\_timezones:
\textgreater{}\textgreater{}\textgreater{}     print(tz)
\end{notice}

\end{fulllineitems}

\index{mod() (astroobs.ObservatoryList method)}

\begin{fulllineitems}
\phantomsection\label{astroobs:astroobs.ObservatoryList.mod}\pysiglinewithargsret{\bfcode{mod}}{\emph{obsid}, \emph{name}, \emph{long}, \emph{lat}, \emph{elevation}, \emph{timezone}, \emph{temp=15.0}, \emph{pressure=1010.0}, \emph{moonAvoidRadius=0.25}, \emph{**kwargs}}{}
Modifies an observatory in the current observatories database.
\begin{description}
\item[{Args:}] \leavevmode\begin{itemize}
\item {} 
obsid (str): id of the observatory to modify. All other parameters redefine the observatory

\end{itemize}

\item[{Kwargs:}] \leavevmode
See class constructor

\item[{Raises:}] \leavevmode\begin{itemize}
\item {} 
NameError: if the observatory ID does not exist

\item {} 
Exception: if a mandatory input parameter is missing when reloading all observatories

\end{itemize}

\end{description}

\begin{notice}{note}{Note:}
Refer to \code{add()} for details on input parameters
\end{notice}

\end{fulllineitems}

\index{nameList() (astroobs.ObservatoryList method)}

\begin{fulllineitems}
\phantomsection\label{astroobs:astroobs.ObservatoryList.nameList}\pysiglinewithargsret{\bfcode{nameList}}{}{}
Provides a list of tuples (obs id, observatory name) in the alphabetical order of the column `observatory name'.

\end{fulllineitems}

\index{rem() (astroobs.ObservatoryList method)}

\begin{fulllineitems}
\phantomsection\label{astroobs:astroobs.ObservatoryList.rem}\pysiglinewithargsret{\bfcode{rem}}{\emph{obsid}, \emph{**kwargs}}{}
Removes an observatory from the current observatories database.
\begin{description}
\item[{Args:}] \leavevmode\begin{itemize}
\item {} 
obsid (str): id of the observatory to remove

\end{itemize}

\item[{Kwargs:}] \leavevmode
See class constructor

\item[{Raises:}] \leavevmode\begin{itemize}
\item {} 
NameError: if the observatory ID does not exist

\item {} 
Exception: if a mandatory input parameter is missing when reloading all observatories

\end{itemize}

\end{description}

\end{fulllineitems}


\end{fulllineitems}

\index{Observatory (class in astroobs)}

\begin{fulllineitems}
\phantomsection\label{astroobs:astroobs.Observatory}\pysiglinewithargsret{\strong{class }\code{astroobs.}\bfcode{Observatory}}{\emph{obs}, \emph{long=None}, \emph{lat=None}, \emph{elevation=None}, \emph{timezone=None}, \emph{temp=None}, \emph{pressure=None}, \emph{moonAvoidRadius=None}, \emph{local\_date=None}, \emph{ut\_date=None}, \emph{horizon\_obs=None}, \emph{dataFile=None}, \emph{epoch=`2000'}, \emph{**kwargs}}{}
Bases: \code{ephem.Observer}, \code{object}

Defines an observatory from which the ephemeris of the twilights or a night-sky target are processed. The \emph{night-time} is base on the given date. It ends at the next sunrise and starts at the sunset preceeding this next sunrise.
\begin{description}
\item[{Args:}] \leavevmode\begin{itemize}
\item {} 
obs (str): id of the observatory to pick from the observatories database OR the name of the custom observatory (in that case, \code{long}, \code{lat}, \code{elevation}, \code{timezone} must also be given, \code{temp}, \code{pressure}, \code{moonAvoidRadius} are optional)

\item {} 
local\_date (see below): the date of observation in local time

\item {} 
ut\_date (see below): the date of observation in UT time

\item {} 
horizon\_obs (float - degrees): minimum altitude at which a target can be observed, default is 30 degrees altitude

\item {} 
epoch (str): the `YYYY' year in which all ra-dec coordinates are converted

\end{itemize}

\item[{Kwargs:}] \leavevmode\begin{itemize}
\item {} 
raiseError (bool): if \code{True}, errors will be raised; if \code{False}, they will be printed. Default is \code{False}

\item {} 
fig: TBD

\end{itemize}

\item[{Raises:}] \leavevmode\begin{itemize}
\item {} 
NameError: if a mandatory input parameter is missing

\item {} 
KeyError: if the observatory ID does not exist

\item {} 
KeyError: if the twilight keyword is unknown

\item {} 
Exception: if the observatory object has no date

\end{itemize}

\end{description}

\begin{notice}{note}{Note:}\begin{itemize}
\item {} 
For details on \code{local\_date} and \code{ut\_date}, refer to \code{Observatory.upd\_date()}

\item {} 
For details on other input parameters, refer to \code{ObservatoryList.add()}

\item {} 
The \code{Observatory} automatically creates and manages a \code{Moon} target under \code{moon} attribute

\item {} 
If \code{obs} is the id of an observatory to pick in the database, the user can still provide \code{temp}, \code{pressure}, \code{moonAvoidRadius} attributes which will override the database default values

\item {} 
\code{horizon} attribute is in radian

\end{itemize}
\end{notice}
\begin{description}
\item[{Main attributes:}] \leavevmode\begin{itemize}
\item {} 
\code{localnight}: gives the local midnight time in local time (YYYY, MM, DD, 23, 59, 59)

\item {} 
\code{date}: gives the local midnight time in UT time

\item {} 
\code{dates}: is a vector of Dublin Julian Dates. Refer to \code{process\_obs()}

\item {} 
\code{lst}: the local sidereal time corresponding to each \code{dates} element

\item {} 
\code{localTimeOffest}: gives the shift in days between UT and local time: local=UT+localTimeOffest

\item {} 
\code{moon}: points to the \code{Moon} target processed for the given observatory and date

\end{itemize}

\item[{Twilight attributes:}] \leavevmode\begin{itemize}
\item {} 
For the next three attributes, \code{XXX} shall be replaced by \{`' (blank), `civil', `nautical', `astro'\} for, respectively, horizon, -6, -12, and -18 degrees altitude

\item {} 
\code{sunriseXXX}: gives the sunrise time for different twilights, in Dublin Julian Dates. e.g.: \code{observatory.sunrise}

\item {} 
\code{sunsetXXX}: gives the sunset time for different twilights, in Dublin Julian Dates. e.g.: \code{observatory.sunsetcivil}

\item {} 
\code{len\_nightXXX}: gives the night duration for different twilights (between corresponding sunset and sunrise), in hours. e.g.: \code{observatory.len\_nightnautical}

\end{itemize}

\end{description}

\begin{notice}{warning}{Warning:}\begin{itemize}
\item {} 
it can occur that the Sun, the Moon or a target does not rise or set for an observatory/date combination. In that case, the corresponding attributes will be set to \code{None}

\end{itemize}
\end{notice}

\begin{Verbatim}[commandchars=\\\{\}]
\PYG{g+gp}{\PYGZgt{}\PYGZgt{}\PYGZgt{} }\PYG{k+kn}{import} \PYG{n+nn}{astroobs.obs} \PYG{k+kn}{as} \PYG{n+nn}{obs}
\PYG{g+gp}{\PYGZgt{}\PYGZgt{}\PYGZgt{} }\PYG{n}{o} \PYG{o}{=} \PYG{n}{obs}\PYG{o}{.}\PYG{n}{Observatory}\PYG{p}{(}\PYG{l+s}{\PYGZsq{}}\PYG{l+s}{ohp}\PYG{l+s}{\PYGZsq{}}\PYG{p}{,} \PYG{n}{local\PYGZus{}date}\PYG{o}{=}\PYG{p}{(}\PYG{l+m+mi}{2015}\PYG{p}{,}\PYG{l+m+mi}{3}\PYG{p}{,}\PYG{l+m+mi}{31}\PYG{p}{,}\PYG{l+m+mi}{23}\PYG{p}{,}\PYG{l+m+mi}{59}\PYG{p}{,}\PYG{l+m+mi}{59}\PYG{p}{)}\PYG{p}{)}
\PYG{g+gp}{\PYGZgt{}\PYGZgt{}\PYGZgt{} }\PYG{n}{o}
\PYG{g+go}{\PYGZlt{}ephem.Observer date=\PYGZsq{}2015/3/31 21:59:59\PYGZsq{} epoch=\PYGZsq{}2000/1/1 12:00:00\PYGZsq{}}
\PYG{g+go}{lon=5:42:48.0 lat=43:55:51.0 elevation=650.0m horizon=\PYGZhy{}0:49:04.8}
\PYG{g+go}{temp=15.0C pressure=1010.0mBar\PYGZgt{}}
\PYG{g+gp}{\PYGZgt{}\PYGZgt{}\PYGZgt{} }\PYG{n}{o}\PYG{o}{.}\PYG{n}{moon}
\PYG{g+go}{Moon \PYGZhy{} phase: 89.2\PYGZpc{}}
\PYG{g+gp}{\PYGZgt{}\PYGZgt{}\PYGZgt{} }\PYG{k}{print}\PYG{p}{(}\PYG{n}{o}\PYG{o}{.}\PYG{n}{sunset}\PYG{p}{,} \PYG{l+s}{\PYGZsq{}}\PYG{l+s}{...}\PYG{l+s}{\PYGZsq{}}\PYG{p}{,} \PYG{n}{o}\PYG{o}{.}\PYG{n}{sunrise}\PYG{p}{,} \PYG{l+s}{\PYGZsq{}}\PYG{l+s}{...}\PYG{l+s}{\PYGZsq{}}\PYG{p}{,} \PYG{n}{o}\PYG{o}{.}\PYG{n}{len\PYGZus{}night}\PYG{p}{)}
\PYG{g+go}{2015/3/31 18:08:40 ... 2015/4/1 05:13:09 ... 11.0746939826}
\PYG{g+gp}{\PYGZgt{}\PYGZgt{}\PYGZgt{} }\PYG{k+kn}{import} \PYG{n+nn}{ephem} \PYG{k+kn}{as} \PYG{n+nn}{E}
\PYG{g+gp}{\PYGZgt{}\PYGZgt{}\PYGZgt{} }\PYG{k}{print}\PYG{p}{(}\PYG{n}{E}\PYG{o}{.}\PYG{n}{Date}\PYG{p}{(}\PYG{n}{o}\PYG{o}{.}\PYG{n}{sunsetastro}\PYG{o}{+}\PYG{n}{o}\PYG{o}{.}\PYG{n}{localTimeOffest}\PYG{p}{)}\PYG{p}{,} \PYG{l+s}{\PYGZsq{}}\PYG{l+s}{...}\PYG{l+s}{\PYGZsq{}}\PYG{p}{,} \PYG{n}{E}\PYG{o}{.}\PYG{n}{Date}\PYG{p}{(}
\PYG{g+go}{        o.sunriseastro+o.localTimeOffest), \PYGZsq{}...\PYGZsq{}, o.len\PYGZus{}nightastro)}
\PYG{g+go}{2015/3/31 21:43:28 ... 2015/4/1 05:38:26 ... 7.91603336949}
\end{Verbatim}
\index{nowArg (astroobs.Observatory attribute)}

\begin{fulllineitems}
\phantomsection\label{astroobs:astroobs.Observatory.nowArg}\pysigline{\bfcode{nowArg}}
Returns the index of \emph{now} in the \code{observatory.dates} vector, or None if \emph{now} is out of its bounds (meaning the observation is not taking place now)

\begin{Verbatim}[commandchars=\\\{\}]
\PYG{g+gp}{\PYGZgt{}\PYGZgt{}\PYGZgt{} }\PYG{k+kn}{import} \PYG{n+nn}{astroobs.obs} \PYG{k+kn}{as} \PYG{n+nn}{obs}
\PYG{g+gp}{\PYGZgt{}\PYGZgt{}\PYGZgt{} }\PYG{k+kn}{import} \PYG{n+nn}{ephem} \PYG{k+kn}{as} \PYG{n+nn}{E}
\PYG{g+gp}{\PYGZgt{}\PYGZgt{}\PYGZgt{} }\PYG{n}{o} \PYG{o}{=} \PYG{n}{obs}\PYG{o}{.}\PYG{n}{Observatory}\PYG{p}{(}\PYG{l+s}{\PYGZsq{}}\PYG{l+s}{ohp}\PYG{l+s}{\PYGZsq{}}\PYG{p}{)}
\PYG{g+gp}{\PYGZgt{}\PYGZgt{}\PYGZgt{} }\PYG{n}{plt}\PYG{o}{.}\PYG{n}{plot}\PYG{p}{(}\PYG{n}{o}\PYG{o}{.}\PYG{n}{dates}\PYG{p}{,} \PYG{n}{o}\PYG{o}{.}\PYG{n}{moon}\PYG{o}{.}\PYG{n}{alt}\PYG{p}{,} \PYG{l+s}{\PYGZsq{}}\PYG{l+s}{k\PYGZhy{}}\PYG{l+s}{\PYGZsq{}}\PYG{p}{)}
\PYG{g+gp}{\PYGZgt{}\PYGZgt{}\PYGZgt{} }\PYG{n}{now} \PYG{o}{=} \PYG{n}{o}\PYG{o}{.}\PYG{n}{nowArg}
\PYG{g+gp}{\PYGZgt{}\PYGZgt{}\PYGZgt{} }\PYG{k}{if} \PYG{n}{now} \PYG{o+ow}{is} \PYG{o+ow}{not} \PYG{n+nb+bp}{None}\PYG{p}{:}
\PYG{g+gp}{\PYGZgt{}\PYGZgt{}\PYGZgt{} }    \PYG{n}{plt}\PYG{o}{.}\PYG{n}{plot}\PYG{p}{(}\PYG{n}{o}\PYG{o}{.}\PYG{n}{dates}\PYG{p}{[}\PYG{n}{now}\PYG{p}{]}\PYG{p}{,} \PYG{n}{o}\PYG{o}{.}\PYG{n}{moon}\PYG{o}{.}\PYG{n}{alt}\PYG{p}{[}\PYG{n}{now}\PYG{p}{]}\PYG{p}{,} \PYG{l+s}{\PYGZsq{}}\PYG{l+s}{ro}\PYG{l+s}{\PYGZsq{}}\PYG{p}{)}
\PYG{g+gp}{\PYGZgt{}\PYGZgt{}\PYGZgt{} }\PYG{k}{else}\PYG{p}{:}
\PYG{g+gp}{\PYGZgt{}\PYGZgt{}\PYGZgt{} }    \PYG{n}{plt}\PYG{o}{.}\PYG{n}{plot}\PYG{p}{(}\PYG{p}{[}\PYG{n}{E}\PYG{o}{.}\PYG{n}{now}\PYG{p}{(}\PYG{p}{)}\PYG{p}{,} \PYG{n}{E}\PYG{o}{.}\PYG{n}{now}\PYG{p}{(}\PYG{p}{)}\PYG{p}{]}\PYG{p}{,} \PYG{p}{[}\PYG{n}{o}\PYG{o}{.}\PYG{n}{moon}\PYG{o}{.}\PYG{n}{alt}\PYG{o}{.}\PYG{n}{min}\PYG{p}{(}\PYG{p}{)}\PYG{p}{,}\PYG{n}{o}\PYG{o}{.}\PYG{n}{moon}\PYG{o}{.}\PYG{n}{alt}\PYG{o}{.}\PYG{n}{max}\PYG{p}{(}\PYG{p}{)}\PYG{p}{]}\PYG{p}{,} \PYG{l+s}{\PYGZsq{}}\PYG{l+s}{r\PYGZhy{}\PYGZhy{}}\PYG{l+s}{\PYGZsq{}}\PYG{p}{)}
\end{Verbatim}

\end{fulllineitems}

\index{plot() (astroobs.Observatory method)}

\begin{fulllineitems}
\phantomsection\label{astroobs:astroobs.Observatory.plot}\pysiglinewithargsret{\bfcode{plot}}{\emph{**kwargs}}{}
Plots the observatory diagram
\begin{description}
\item[{Kwargs:}] \leavevmode\begin{itemize}
\item {} 
See class constructor

\item {} 
dt (float - hour): the spacing of x-axis labels, default is 1 hour (not with polar mode)

\item {} 
t0 (float - DJD or {[}0-24{]}): the date of the first tick-label of x-axis, default is sunsetastro. The time type must correspond to \code{time} parameter (not with polar mode)

\item {} 
xlim ({[}xmin, xmax{]}): bounds for x-axis, default is full night span (not with polar mode)

\item {} 
retxdisp (bool): if \code{True}, bounds of x-axis displayed values are returned (\code{xdisp} key)

\item {} 
ylim ({[}ymin, ymax{]}): bounds for y-axis, default is {[}horizon\_obs-10, 90{]} (not with polar mode)

\item {} 
xlabel (str): label for x-axis, default `Time (UT)'

\item {} 
ylabel (str): label for y-axis, default `Elevation (°)'

\item {} 
title (str): title of the diagram, default is observatory name or coordinates

\item {} 
ymin\_margin (float): margin between xmin of graph and horizon\_obs. Low priority vs ylim, default is 10 (not with polar mode)

\item {} 
retfignum (bool): if \code{True}, the figure number will be returned, default is \code{False}

\item {} 
fignum (int): figure number on which to plot, default is \code{False}

\item {} 
retaxnum (bool): if \code{True}, the ax index as in \code{figure.axes{[}n{]}} will be returned, default is \code{False}

\item {} 
axnum (int): axes index on which to plot, default is \code{None} (create new ax)

\item {} 
retfig (bool): if \code{True}, the figure object will be returned, default is \code{False}

\item {} 
fig (figure): figure object on which to plot, default is \code{None} (use fignum)

\item {} 
retax (bool): if \code{True}, the ax will be returned, default is \code{False}

\item {} 
ax (axes): ax on which to plot, default is \code{None}

\item {} 
now (bool): if \code{True} and within range, a vertical line as indication of ``now'' will be shown, default is True

\item {} 
retnow (bool): returns the line object (\code{nowline} key) corresponding to the `now-line', default is \code{False}

\item {} 
legend (bool): whether to add a legend or not, default is \code{True}

\item {} 
loc: location of the legend, default is 8 (top right), refer to plt.legend

\item {} 
ncol: number of columns in the legend, default is 3, refer to plt.legend

\item {} 
columnspacing: spacing between columns in the legend, refer to plt.legend

\item {} 
lfs: legend font size, default is 11

\item {} 
textlbl (bool): if \code{True}, a text label with target name or coordinates will be added near transit, default is \code{False}

\item {} 
polar (bool): if \code{True}, plots the sky view, otherwise plots target attribute versus time

\item {} 
time (str): the type of the x-axis time, \code{ut} for UT, \code{loc} for local time and \code{lst} {[}0-24{]} for local sidereal time, default is \code{ut} (not with polar mode)

\end{itemize}

\item[{Raises:}] \leavevmode
N/A

\end{description}

\end{fulllineitems}

\index{process\_obs() (astroobs.Observatory method)}

\begin{fulllineitems}
\phantomsection\label{astroobs:astroobs.Observatory.process_obs}\pysiglinewithargsret{\bfcode{process\_obs}}{\emph{pts=200}, \emph{margin=15}, \emph{fullhour=False}, \emph{**kwargs}}{}
Processes all twilights as well as moon rise, set and position through night for the given observatory and date.
Creates the vector \code{observatory.dates} which is the vector containing all timestamps at which the moon and the targets will be processed.
\begin{description}
\item[{Args:}] \leavevmode\begin{itemize}
\item {} 
pts (int) {[}optional{]}: the size of the \code{dates} vector, whose elements are linearly spaced in time

\item {} 
margin (float - minutes) {[}optional{]}: the margin between the first element of the vector \code{dates} and the sunset, and between the sunrise and its last element

\item {} 
fullhour (bool) {[}optional{]}: if \code{True}, then the vector \code{dates} will start and finish on the first full hour preceeding sunset and following sunrise

\end{itemize}

\item[{Kwargs:}] \leavevmode
See class constructor

\item[{Raises:}] \leavevmode\begin{itemize}
\item {} 
KeyError: if the twilight keyword is unknown

\item {} 
Exception: if the observatory object has no date

\end{itemize}

\end{description}

\begin{notice}{note}{Note:}
In case the observatory is in polar regions where the sun does not alway set and rise everyday, the first and last elements of the \code{dates} vector are set to local midday right before and after the local midnight of the observation date. e.g.: 24h night centered on the local midnight.
\end{notice}

\end{fulllineitems}

\index{upd\_date() (astroobs.Observatory method)}

\begin{fulllineitems}
\phantomsection\label{astroobs:astroobs.Observatory.upd_date}\pysiglinewithargsret{\bfcode{upd\_date}}{\emph{ut\_date=None}, \emph{local\_date=None}, \emph{force=False}, \emph{**kwargs}}{}
Updates the date of the observatory, and re-process the observatory parameters if the date is different.
\begin{description}
\item[{Args:}] \leavevmode\begin{itemize}
\item {} 
ut\_date (see below): the date of observation in UT time

\item {} 
local\_date (see below): the date of observation in local time

\item {} 
force (bool): if \code{False}, the observatory is re-processed only if the date changed

\end{itemize}

\item[{Kwargs:}] \leavevmode
See class constructor

\item[{Raises:}] \leavevmode\begin{itemize}
\item {} 
KeyError: if the twilight keyword is unknown

\item {} 
Exception: if the observatory object has no date

\end{itemize}

\item[{Returns:}] \leavevmode
\code{True} if the date was changed, otherwise \code{False}

\end{description}

\begin{notice}{note}{Note:}\begin{itemize}
\item {} 
\code{local\_date} and \code{ut\_date} can be date-tuples \code{(yyyy, mm, dd, {[}hh, mm, ss{]})}, timestamps, datetime structures or ephem.Date instances.

\item {} 
If both are given, \code{ut\_date} has higher priority

\item {} 
If neither of those are given, the date is automatically set to \emph{tonight} or \emph{now} (whether the sun has already set or not)

\end{itemize}
\end{notice}

\end{fulllineitems}


\end{fulllineitems}

\index{Target (class in astroobs)}

\begin{fulllineitems}
\phantomsection\label{astroobs:astroobs.Target}\pysiglinewithargsret{\strong{class }\code{astroobs.}\bfcode{Target}}{\emph{ra}, \emph{dec}, \emph{name}, \emph{input\_epoch=`2000'}, \emph{obs=None}, \emph{**kwargs}}{}
Bases: \code{object}

Initialises a target object from its right ascension and declination. Optionaly, processes the target for the observatory and date given (refer to \code{Target.process()}).
\begin{description}
\item[{Args:}] \leavevmode\begin{itemize}
\item {} 
ra (str `hh:mm:ss.s' or float - degrees): the right ascension of the target

\item {} 
dec (str `+/-dd:mm:ss.s' or float - degrees): the declination of the target

\item {} 
name (str): the name of the target, for display

\item {} 
obs (\code{Observatory}) {[}optional{]}: the observatory for which to process the target

\item {} 
input\_epoch (str): the `YYYY' year of epoch in which the ra-dec coordinates are given. These coordinates will corrected with precession if the epoch of observatory is different

\end{itemize}

\item[{Kwargs:}] \leavevmode\begin{itemize}
\item {} 
raiseError (bool): if \code{True}, errors will be raised; if \code{False}, they will be printed. Default is \code{False}

\end{itemize}

\item[{Raises:}] \leavevmode
N/A

\end{description}
\index{dec (astroobs.Target attribute)}

\begin{fulllineitems}
\phantomsection\label{astroobs:astroobs.Target.dec}\pysigline{\bfcode{dec}}
The declination of the target, displayed as tuple (+/-dd, mm, ss)

\end{fulllineitems}

\index{decStr (astroobs.Target attribute)}

\begin{fulllineitems}
\phantomsection\label{astroobs:astroobs.Target.decStr}\pysigline{\bfcode{decStr}}
A pretty printable version of the declination of the target

\end{fulllineitems}

\index{plot() (astroobs.Target method)}

\begin{fulllineitems}
\phantomsection\label{astroobs:astroobs.Target.plot}\pysiglinewithargsret{\bfcode{plot}}{\emph{obs}, \emph{y='alt'}, \emph{**kwargs}}{}
Plots the y-parameter vs time diagram for the target at the given observatory and date
\begin{description}
\item[{Args:}] \leavevmode\begin{itemize}
\item {} 
obs (\code{Observatory}): the observatory for which to plot the target

\item {} 
y (object attribute): the y-data to plot

\end{itemize}

\item[{Kwargs:}] \leavevmode\begin{itemize}
\item {} 
See class constructor

\item {} 
See \code{Observatory.plot()}

\item {} 
simpleplt (bool): if \code{True}, the observatory plot will not be plotted, default is \code{False}

\item {} 
color (str or \#XXXXXX): the color of the target curve, default is `k'

\item {} 
lw (float): the linewidth, default is 1

\end{itemize}

\item[{Raises:}] \leavevmode
N/A

\end{description}

\end{fulllineitems}

\index{polar() (astroobs.Target method)}

\begin{fulllineitems}
\phantomsection\label{astroobs:astroobs.Target.polar}\pysiglinewithargsret{\bfcode{polar}}{\emph{obs}, \emph{**kwargs}}{}
Plots the sky-view diagram for the target at the given observatory and date
\begin{description}
\item[{Args:}] \leavevmode\begin{itemize}
\item {} 
obs (\code{Observatory}): the observatory for which to plot the target

\item {} 
y (object attribute): the y-data to plot

\end{itemize}

\item[{Kwargs:}] \leavevmode\begin{itemize}
\item {} 
See class constructor

\item {} 
See \code{Observatory.plot()}

\item {} 
See \code{Target.plot()}

\end{itemize}

\item[{Raises:}] \leavevmode
N/A

\end{description}

\end{fulllineitems}

\index{process() (astroobs.Target method)}

\begin{fulllineitems}
\phantomsection\label{astroobs:astroobs.Target.process}\pysiglinewithargsret{\bfcode{process}}{\emph{obs}, \emph{**kwargs}}{}
Processes the target for the given observatory and date.
\begin{description}
\item[{Args:}] \leavevmode\begin{itemize}
\item {} 
obs (\code{Observatory}): the observatory for which to process the target

\end{itemize}

\item[{Kwargs:}] \leavevmode
See class constructor

\item[{Raises:}] \leavevmode
N/A

\item[{Creates vector attributes:}] \leavevmode\begin{itemize}
\item {} 
\code{airmass}: the airmass of the target

\item {} 
\code{ha}: the hour angle of the target (degrees)

\item {} 
\code{alt}: the altitude of the target (degrees - horizon is 0)

\item {} 
\code{az}: the azimuth of the target (degrees)

\item {} 
\code{moondist}: the angular distance between the moon and the target (degrees)

\end{itemize}

\end{description}

\begin{notice}{note}{Note:}\begin{itemize}
\item {} 
All previous attributes are vectors related to the time vector of the observatory used for processing, stored under \code{dates} attribute

\end{itemize}
\end{notice}
\begin{description}
\item[{Other attributes:}] \leavevmode\begin{itemize}
\item {} 
\code{rise\_time}, \code{rise\_az}: the time (ephem.Date) and the azimuth (degree) of the rise of the target

\item {} 
\code{set\_time}, \code{set\_az}: the time (ephem.Date) and the azimuth (degree) of the setting of the target

\item {} 
\code{transit\_time}, \code{transit\_az}: the time (ephem.Date) and the azimuth (degree) of the transit of the target

\end{itemize}

\end{description}

\begin{notice}{warning}{Warning:}\begin{itemize}
\item {} 
it can occur that the target does not rise or set for an observatory/date combination. In that case, the corresponding attributes will be set to \code{None}, i.e. \code{set\_time}, \code{set\_az}, \code{rise\_time}, \code{rise\_az}. In that case, an additional parameter is added to the Target object: \code{Target.alwaysUp} which is \code{True} if the target never sets and \code{False} if it never rises above the horizon.

\end{itemize}
\end{notice}

\end{fulllineitems}

\index{ra (astroobs.Target attribute)}

\begin{fulllineitems}
\phantomsection\label{astroobs:astroobs.Target.ra}\pysigline{\bfcode{ra}}
The right ascension of the target, displayed as tuple (hh, mm, ss)

\end{fulllineitems}

\index{raStr (astroobs.Target attribute)}

\begin{fulllineitems}
\phantomsection\label{astroobs:astroobs.Target.raStr}\pysigline{\bfcode{raStr}}
A pretty printable version of the right ascension of the target

\end{fulllineitems}

\index{whenobs() (astroobs.Target method)}

\begin{fulllineitems}
\phantomsection\label{astroobs:astroobs.Target.whenobs}\pysiglinewithargsret{\bfcode{whenobs}}{\emph{obs}, \emph{fromDate='now'}, \emph{toDate='now+30day'}, \emph{plot=True}, \emph{ret=False}, \emph{dday=1}, \emph{**kwargs}}{}
Processes the target for the given observatory and dat.
\begin{description}
\item[{Args:}] \leavevmode\begin{itemize}
\item {} 
obs (\code{Observatory}): the observatory for which to process the target

\item {} 
fromDate (see below): the start date of the range

\item {} 
toDate (see below): the end date of the range

\item {} 
plot: whether it plots the diagram

\item {} 
ret: whether it returns the values

\item {} 
dday: the

\end{itemize}

\item[{Kwargs:}] \leavevmode
See class constructor
* legend (bool): whether to add a legend or not, default is \code{True}
* loc: location of the legend, default is 8 (top right), refer to plt.legend
* ncol: number of columns in the legend, default is 3, refer to plt.legend
* columnspacing: spacing between columns in the legend, refer to plt.legend
* lfs: legend font size, default is 11

\item[{Raises:}] \leavevmode
N/A

\end{description}

\begin{notice}{note}{Note:}\begin{itemize}
\item {} 
\code{local\_date} and \code{ut\_date} can be date-tuples \code{(yyyy, mm, dd, {[}hh, mm, ss{]})}, timestamps, datetime structures or ephem.Date instances.

\end{itemize}
\end{notice}

\end{fulllineitems}


\end{fulllineitems}

\index{Moon (class in astroobs)}

\begin{fulllineitems}
\phantomsection\label{astroobs:astroobs.Moon}\pysiglinewithargsret{\strong{class }\code{astroobs.}\bfcode{Moon}}{\emph{obs=None}, \emph{input\_epoch=`2000'}, \emph{**kwargs}}{}
Bases: \code{astroobs.Target.Target}

Initialises the Moon. Optionaly, processes the Moon for the observatory and date given (refer to \code{Moon.process()}).
\begin{description}
\item[{Args:}] \leavevmode\begin{itemize}
\item {} 
obs (\code{Observatory}) {[}optional{]}: the observatory for which to process the Moon

\end{itemize}

\item[{Kwargs:}] \leavevmode\begin{itemize}
\item {} 
raiseError (bool): if \code{True}, errors will be raised; if \code{False}, they will be printed. Default is \code{False}

\end{itemize}

\item[{Raises:}] \leavevmode
N/A

\end{description}
\index{dec (astroobs.Moon attribute)}

\begin{fulllineitems}
\phantomsection\label{astroobs:astroobs.Moon.dec}\pysigline{\bfcode{dec}}
The declination of the Moon, displayed as tuple of np.array (+/-dd, mm, ss)

\end{fulllineitems}

\index{decStr (astroobs.Moon attribute)}

\begin{fulllineitems}
\phantomsection\label{astroobs:astroobs.Moon.decStr}\pysigline{\bfcode{decStr}}
A pretty printable version of the mean of the declination of the moon

\end{fulllineitems}

\index{plot() (astroobs.Moon method)}

\begin{fulllineitems}
\phantomsection\label{astroobs:astroobs.Moon.plot}\pysiglinewithargsret{\bfcode{plot}}{\emph{obs}, \emph{y='alt'}, \emph{**kwargs}}{}
Plots the y-parameter vs time diagram for the moon at the given observatory and date
\begin{description}
\item[{Args:}] \leavevmode\begin{itemize}
\item {} 
obs (\code{Observatory}): the observatory for which to plot the moon

\end{itemize}

\item[{Kwargs:}] \leavevmode\begin{itemize}
\item {} 
See class constructor

\item {} 
See \code{Observatory.plot()}

\item {} 
See \code{Target.plot()}

\end{itemize}

\item[{Raises:}] \leavevmode
N/A

\end{description}

\end{fulllineitems}

\index{polar() (astroobs.Moon method)}

\begin{fulllineitems}
\phantomsection\label{astroobs:astroobs.Moon.polar}\pysiglinewithargsret{\bfcode{polar}}{\emph{obs}, \emph{**kwargs}}{}
Plots the y-parameter vs time diagram for the moon at the given observatory and date
\begin{description}
\item[{Args:}] \leavevmode\begin{itemize}
\item {} 
obs (\code{Observatory}): the observatory for which to plot the moon

\end{itemize}

\item[{Kwargs:}] \leavevmode\begin{itemize}
\item {} 
See class constructor

\item {} 
See \code{Observatory.plot()}

\item {} 
See \code{Target.plot()}

\end{itemize}

\item[{Raises:}] \leavevmode
N/A

\end{description}

\end{fulllineitems}

\index{process() (astroobs.Moon method)}

\begin{fulllineitems}
\phantomsection\label{astroobs:astroobs.Moon.process}\pysiglinewithargsret{\bfcode{process}}{\emph{obs}, \emph{**kwargs}}{}
Processes the moon for the given observatory and date.
\begin{description}
\item[{Args:}] \leavevmode\begin{itemize}
\item {} 
obs (\code{Observatory}): the observatory for which to process the moon

\end{itemize}

\item[{Kwargs:}] \leavevmode
See class constructor

\item[{Raises:}] \leavevmode
N/A

\item[{Creates vector attributes:}] \leavevmode\begin{itemize}
\item {} 
\code{airmass}: the airmass of the moon

\item {} 
\code{ha}: the hour angle of the moon (degrees)

\item {} 
\code{alt}: the altitude of the moon (degrees - horizon is 0)

\item {} 
\code{az}: the azimuth of the moon (degrees)

\item {} 
\code{ra}: the right ascension of the moon, see \code{Moon.ra()}

\item {} 
\code{dec}: the declination of the moon, see \code{Moon.dec()}

\end{itemize}

\end{description}

\begin{notice}{note}{Note:}\begin{itemize}
\item {} 
All previous attributes are vectors related to the time vector of the observatory used for processing: \code{obs.dates}

\end{itemize}
\end{notice}
\begin{description}
\item[{Other attributes:}] \leavevmode\begin{itemize}
\item {} 
\code{rise\_time}, \code{rise\_az}: the time (ephem.Date) and the azimuth (degree) of the rise of the moon

\item {} 
\code{set\_time}, \code{set\_az}: the time (ephem.Date) and the azimuth (degree) of the setting of the moon

\item {} 
\code{transit\_time}, \code{transit\_az}: the time (ephem.Date) and the azimuth (degree) of the transit of the moon

\end{itemize}

\end{description}

\begin{notice}{warning}{Warning:}\begin{itemize}
\item {} 
it can occur that the moon does not rise or set for an observatory/date combination. In that case, the corresponding attributes will be set to \code{None}, i.e. \code{set\_time}, \code{set\_az}, \code{rise\_time}, \code{rise\_az}. In that case, an additional parameter is added to the Moon object: \code{Moon.alwaysUp} which is \code{True} if the Moon never sets and \code{False} if it never rises above the horizon.

\end{itemize}
\end{notice}

\end{fulllineitems}

\index{ra (astroobs.Moon attribute)}

\begin{fulllineitems}
\phantomsection\label{astroobs:astroobs.Moon.ra}\pysigline{\bfcode{ra}}
The right ascension of the Moon, displayed as tuple of np.array (hh, mm, ss)

\end{fulllineitems}

\index{raStr (astroobs.Moon attribute)}

\begin{fulllineitems}
\phantomsection\label{astroobs:astroobs.Moon.raStr}\pysigline{\bfcode{raStr}}
A pretty printable version of the mean of the right ascension of the moon

\end{fulllineitems}


\end{fulllineitems}

\index{TargetSIMBAD (class in astroobs)}

\begin{fulllineitems}
\phantomsection\label{astroobs:astroobs.TargetSIMBAD}\pysiglinewithargsret{\strong{class }\code{astroobs.}\bfcode{TargetSIMBAD}}{\emph{name}, \emph{obs=None}, \emph{input\_epoch=`2000'}, \emph{**kwargs}}{}
Bases: \code{astroobs.Target.Target}

Initialises a target object from an online SIMBAD database name-search. Optionaly, processes the target for the observatory and date given (refer to \code{TargetSIMBAD.process()}).
\begin{description}
\item[{Args:}] \leavevmode\begin{itemize}
\item {} 
name (str): the name of the target as if performing an online SIMBAD search

\item {} 
obs (\code{Observatory}) {[}optional{]}: the observatory for which to process the target

\end{itemize}

\item[{Kwargs:}] \leavevmode\begin{itemize}
\item {} 
raiseError (bool): if \code{True}, errors will be raised; if \code{False}, they will be printed. Default is \code{False}

\end{itemize}

\item[{Raises:}] \leavevmode
N/A

\item[{Creates attributes:}] \leavevmode\begin{itemize}
\item {} 
\code{flux}: a dictionary of the magnitudes of the target. Keys are part or all of {[}'U','B','V','R','I','J','H','K'{]}

\item {} 
\code{link}: the link to paste into a web-browser to display the SIMBAD page of the target

\item {} 
\code{linkbib}: the link to paste into a web-browser to display the references on the SIMBAD page of the target

\item {} 
\code{hd}: if applicable, the HD number of the target

\item {} 
\code{hr}: if applicable, the HR number of the target

\item {} 
\code{hip}: if applicable, the HIP number of the target

\end{itemize}

\end{description}

\end{fulllineitems}

\index{Observation (class in astroobs)}

\begin{fulllineitems}
\phantomsection\label{astroobs:astroobs.Observation}\pysiglinewithargsret{\strong{class }\code{astroobs.}\bfcode{Observation}}{\emph{obs}, \emph{long=None}, \emph{lat=None}, \emph{elevation=None}, \emph{timezone=None}, \emph{temp=None}, \emph{pressure=None}, \emph{moonAvoidRadius=None}, \emph{local\_date=None}, \emph{ut\_date=None}, \emph{horizon\_obs=None}, \emph{dataFile=None}, \emph{epoch=`2000'}, \emph{**kwargs}}{}
Bases: \code{astroobs.Observatory.Observatory}

Assembles together an \code{Observatory} (including itself the \code{Moon} target), and a list of \code{Target}.
\begin{description}
\item[{For use and docs refer to:}] \leavevmode\begin{itemize}
\item {} 
\code{add\_target()} to add a target to the list

\item {} 
\code{rem\_target()} to remove one

\item {} 
\code{change\_obs()} to change the observatory

\item {} 
\code{change\_date()} to change the date of observation

\end{itemize}

\item[{Kwargs:}] \leavevmode\begin{itemize}
\item {} 
raiseError (bool): if \code{True}, errors will be raised; if \code{False}, they will be printed. Default is \code{False}

\item {} 
fig: TBD

\end{itemize}

\item[{Raises:}] \leavevmode
See \code{Observatory}

\end{description}

\begin{notice}{warning}{Warning:}\begin{itemize}
\item {} 
it can occur that the Sun, the Moon or a target does not rise or set for an observatory/date combination. In that case, the corresponding attributes will be set to \code{None}

\end{itemize}
\end{notice}

\begin{Verbatim}[commandchars=\\\{\}]
\PYG{g+gp}{\PYGZgt{}\PYGZgt{}\PYGZgt{} }\PYG{k+kn}{import} \PYG{n+nn}{astroobs.obs} \PYG{k+kn}{as} \PYG{n+nn}{obs}
\PYG{g+gp}{\PYGZgt{}\PYGZgt{}\PYGZgt{} }\PYG{n}{o} \PYG{o}{=} \PYG{n}{obs}\PYG{o}{.}\PYG{n}{Observation}\PYG{p}{(}\PYG{l+s}{\PYGZsq{}}\PYG{l+s}{ohp}\PYG{l+s}{\PYGZsq{}}\PYG{p}{,} \PYG{n}{local\PYGZus{}date}\PYG{o}{=}\PYG{p}{(}\PYG{l+m+mi}{2015}\PYG{p}{,}\PYG{l+m+mi}{3}\PYG{p}{,}\PYG{l+m+mi}{31}\PYG{p}{,}\PYG{l+m+mi}{23}\PYG{p}{,}\PYG{l+m+mi}{59}\PYG{p}{,}\PYG{l+m+mi}{59}\PYG{p}{)}\PYG{p}{)}
\PYG{g+gp}{\PYGZgt{}\PYGZgt{}\PYGZgt{} }\PYG{n}{o}
\PYG{g+go}{Observation at Observatoire de Haute Provence on 2015/6/21\PYGZhy{}22. 0 targets.}
\PYG{g+go}{    Moon phase: 89.2\PYGZpc{}}
\PYG{g+gp}{\PYGZgt{}\PYGZgt{}\PYGZgt{} }\PYG{n}{o}\PYG{o}{.}\PYG{n}{moon}
\PYG{g+go}{Moon \PYGZhy{} phase: 89.2\PYGZpc{}}
\PYG{g+gp}{\PYGZgt{}\PYGZgt{}\PYGZgt{} }\PYG{k}{print} \PYG{n}{o}\PYG{o}{.}\PYG{n}{sunset}\PYG{p}{,} \PYG{l+s}{\PYGZsq{}}\PYG{l+s}{...}\PYG{l+s}{\PYGZsq{}}\PYG{p}{,} \PYG{n}{o}\PYG{o}{.}\PYG{n}{sunrise}\PYG{p}{,} \PYG{l+s}{\PYGZsq{}}\PYG{l+s}{...}\PYG{l+s}{\PYGZsq{}}\PYG{p}{,} \PYG{n}{o}\PYG{o}{.}\PYG{n}{len\PYGZus{}night}
\PYG{g+go}{2015/3/31 18:08:40 ... 2015/4/1 05:13:09 ... 11.0746939826}
\PYG{g+gp}{\PYGZgt{}\PYGZgt{}\PYGZgt{} }\PYG{k+kn}{import} \PYG{n+nn}{ephem} \PYG{k+kn}{as} \PYG{n+nn}{E}
\PYG{g+gp}{\PYGZgt{}\PYGZgt{}\PYGZgt{} }\PYG{k}{print}\PYG{p}{(}\PYG{n}{E}\PYG{o}{.}\PYG{n}{Date}\PYG{p}{(}\PYG{n}{o}\PYG{o}{.}\PYG{n}{sunsetastro}\PYG{o}{+}\PYG{n}{o}\PYG{o}{.}\PYG{n}{localTimeOffest}\PYG{p}{)}\PYG{p}{,} \PYG{l+s}{\PYGZsq{}}\PYG{l+s}{...}\PYG{l+s}{\PYGZsq{}}\PYG{p}{,} \PYG{n}{E}\PYG{o}{.}\PYG{n}{Date}\PYG{p}{(}
\PYG{g+go}{        o.sunriseastro+o.localTimeOffest), \PYGZsq{}...\PYGZsq{}, o.len\PYGZus{}nightastro)}
\PYG{g+go}{2015/3/31 21:43:28 ... 2015/4/1 05:38:26 ... 7.91603336949}
\PYG{g+gp}{\PYGZgt{}\PYGZgt{}\PYGZgt{} }\PYG{n}{o}\PYG{o}{.}\PYG{n}{add\PYGZus{}target}\PYG{p}{(}\PYG{l+s}{\PYGZsq{}}\PYG{l+s}{vega}\PYG{l+s}{\PYGZsq{}}\PYG{p}{)}
\PYG{g+gp}{\PYGZgt{}\PYGZgt{}\PYGZgt{} }\PYG{n}{o}\PYG{o}{.}\PYG{n}{add\PYGZus{}target}\PYG{p}{(}\PYG{l+s}{\PYGZsq{}}\PYG{l+s}{mystar}\PYG{l+s}{\PYGZsq{}}\PYG{p}{,} \PYG{n}{dec}\PYG{o}{=}\PYG{l+m+mf}{19.1824}\PYG{p}{,} \PYG{n}{ra}\PYG{o}{=}\PYG{l+m+mf}{213.9153}\PYG{p}{)}
\PYG{g+gp}{\PYGZgt{}\PYGZgt{}\PYGZgt{} }\PYG{n}{o}\PYG{o}{.}\PYG{n}{targets}
\PYG{g+go}{[Target: \PYGZsq{}vega\PYGZsq{}, 18h36m56.3s +38°35\PYGZsq{}8.1\PYGZdq{}, O,}
\PYG{g+go}{ Target: \PYGZsq{}mystar\PYGZsq{}, 14h15m39.7s +19°16\PYGZsq{}43.8\PYGZdq{}, O]}
\PYG{g+gp}{\PYGZgt{}\PYGZgt{}\PYGZgt{} }\PYG{k}{print}\PYG{p}{(}\PYG{l+s}{\PYGZdq{}}\PYG{l+s+si}{\PYGZpc{}s}\PYG{l+s}{ mags: }\PYG{l+s}{\PYGZsq{}}\PYG{l+s}{K}\PYG{l+s}{\PYGZsq{}}\PYG{l+s}{: }\PYG{l+s+si}{\PYGZpc{}2.2f}\PYG{l+s}{, }\PYG{l+s}{\PYGZsq{}}\PYG{l+s}{R}\PYG{l+s}{\PYGZsq{}}\PYG{l+s}{: }\PYG{l+s+si}{\PYGZpc{}2.2f}\PYG{l+s}{\PYGZdq{}}\PYG{o}{\PYGZpc{}}\PYG{p}{(}\PYG{n}{o}\PYG{o}{.}\PYG{n}{targets}\PYG{p}{[}\PYG{l+m+mi}{0}\PYG{p}{]}\PYG{o}{.}\PYG{n}{name}\PYG{p}{,}
\PYG{g+go}{        o.targets[0].flux[\PYGZsq{}K\PYGZsq{}], o.targets[0].flux[\PYGZsq{}R\PYGZsq{}]))}
\PYG{g+go}{vega mags: \PYGZsq{}K\PYGZsq{}: 0.13, \PYGZsq{}R\PYGZsq{}: 0.07}
\end{Verbatim}
\index{add\_target() (astroobs.Observation method)}

\begin{fulllineitems}
\phantomsection\label{astroobs:astroobs.Observation.add_target}\pysiglinewithargsret{\bfcode{add\_target}}{\emph{tgt}, \emph{ra=None}, \emph{dec=None}, \emph{name='`}, \emph{**kwargs}}{}
Adds a target to the observation list
\begin{description}
\item[{Args:}] \leavevmode\begin{itemize}
\item {} 
tgt (see below): the index of the target in the \code{Observation.targets} list

\item {} 
ra (`hh:mm:ss.s' or decimal degree) {[}optional{]}: the right ascension of the target to add to the observation list. See below

\item {} 
dec (`+/-dd:mm:ss.s' or decimal degree) {[}optional{]}: the declination of the target to add to the observation list. See below

\item {} 
name (string) {[}optional{]}: the name of the target to add to the observation list. See below

\end{itemize}

\item[{\code{tgt} arg can be:}] \leavevmode\begin{itemize}
\item {} 
a \code{Target} instance: all other parameters are ignored

\item {} 
a target name (string): if \code{ra} and \code{dec} are not \code{None}, the target is added with the provided coordinates; if \code{None}, a SIMBAD search is performed on \code{tgt}. \code{name} is ignored

\item {} 
a ra-dec string (`hh:mm:ss.s +/-dd:mm:ss.s'): in that case, \code{ra} and \code{dec} will be ignored and \code{name} will be the name of the target

\end{itemize}

\item[{Kwargs:}] \leavevmode
See class constructor

\item[{Raises:}] \leavevmode\begin{itemize}
\item {} 
ValueError: if ra-dec formating was not understood

\end{itemize}

\end{description}

\begin{notice}{note}{Note:}\begin{itemize}
\item {} 
Automatically processes the target for the given observatory and date

\end{itemize}
\end{notice}

\begin{Verbatim}[commandchars=\\\{\}]
\PYG{g+gp}{\PYGZgt{}\PYGZgt{}\PYGZgt{} }\PYG{k+kn}{import} \PYG{n+nn}{astroobs.obs} \PYG{k+kn}{as} \PYG{n+nn}{obs}
\PYG{g+gp}{\PYGZgt{}\PYGZgt{}\PYGZgt{} }\PYG{n}{o} \PYG{o}{=} \PYG{n}{obs}\PYG{o}{.}\PYG{n}{Observation}\PYG{p}{(}\PYG{l+s}{\PYGZsq{}}\PYG{l+s}{ohp}\PYG{l+s}{\PYGZsq{}}\PYG{p}{,} \PYG{n}{local\PYGZus{}date}\PYG{o}{=}\PYG{p}{(}\PYG{l+m+mi}{2015}\PYG{p}{,}\PYG{l+m+mi}{3}\PYG{p}{,}\PYG{l+m+mi}{31}\PYG{p}{,}\PYG{l+m+mi}{23}\PYG{p}{,}\PYG{l+m+mi}{59}\PYG{p}{,}\PYG{l+m+mi}{59}\PYG{p}{)}\PYG{p}{)}
\PYG{g+gp}{\PYGZgt{}\PYGZgt{}\PYGZgt{} }\PYG{n}{arc} \PYG{o}{=} \PYG{n}{obs}\PYG{o}{.}\PYG{n}{TargetSIMBAD}\PYG{p}{(}\PYG{l+s}{\PYGZsq{}}\PYG{l+s}{arcturus}\PYG{l+s}{\PYGZsq{}}\PYG{p}{)}
\PYG{g+gp}{\PYGZgt{}\PYGZgt{}\PYGZgt{} }\PYG{n}{o}\PYG{o}{.}\PYG{n}{add\PYGZus{}target}\PYG{p}{(}\PYG{n}{arc}\PYG{p}{)}
\PYG{g+gp}{\PYGZgt{}\PYGZgt{}\PYGZgt{} }\PYG{n}{o}\PYG{o}{.}\PYG{n}{add\PYGZus{}target}\PYG{p}{(}\PYG{l+s}{\PYGZsq{}}\PYG{l+s}{arcturus}\PYG{l+s}{\PYGZsq{}}\PYG{p}{)}
\PYG{g+gp}{\PYGZgt{}\PYGZgt{}\PYGZgt{} }\PYG{n}{o}\PYG{o}{.}\PYG{n}{add\PYGZus{}target}\PYG{p}{(}\PYG{l+s}{\PYGZsq{}}\PYG{l+s}{arcturusILoveYou}\PYG{l+s}{\PYGZsq{}}\PYG{p}{,} \PYG{n}{dec}\PYG{o}{=}\PYG{l+m+mf}{19.1824}\PYG{p}{,} \PYG{n}{ra}\PYG{o}{=}\PYG{l+m+mf}{213.9153}\PYG{p}{)}
\PYG{g+gp}{\PYGZgt{}\PYGZgt{}\PYGZgt{} }\PYG{n}{o}\PYG{o}{.}\PYG{n}{add\PYGZus{}target}\PYG{p}{(}\PYG{l+s}{\PYGZsq{}}\PYG{l+s}{14:15:39.67 +10:10:56.67}\PYG{l+s}{\PYGZsq{}}\PYG{p}{,} \PYG{n}{name}\PYG{o}{=}\PYG{l+s}{\PYGZsq{}}\PYG{l+s}{arcturus}\PYG{l+s}{\PYGZsq{}}\PYG{p}{)}
\PYG{g+gp}{\PYGZgt{}\PYGZgt{}\PYGZgt{} }\PYG{n}{o}\PYG{o}{.}\PYG{n}{targets} 
\PYG{g+go}{[Target: \PYGZsq{}arcturus\PYGZsq{}, 14h15m39.7s +19°16\PYGZsq{}43.8\PYGZdq{}, O,}
\PYG{g+go}{ Target: \PYGZsq{}arcturus\PYGZsq{}, 14h15m39.7s +19°16\PYGZsq{}43.8\PYGZdq{}, O,}
\PYG{g+go}{ Target: \PYGZsq{}arcturus\PYGZsq{}, 14h15m39.7s +10°40\PYGZsq{}43.8\PYGZdq{}, O,}
\PYG{g+go}{ Target: \PYGZsq{}arcturus\PYGZsq{}, 14h15m39.7s +19°16\PYGZsq{}43.8\PYGZdq{}, O]}
\end{Verbatim}

\end{fulllineitems}

\index{change\_date() (astroobs.Observation method)}

\begin{fulllineitems}
\phantomsection\label{astroobs:astroobs.Observation.change_date}\pysiglinewithargsret{\bfcode{change\_date}}{\emph{ut\_date=None}, \emph{local\_date=None}, \emph{recalcAll=False}, \emph{**kwargs}}{}
Changes the date of the observation and optionaly re-processes targets for the same observatory and new date
\begin{description}
\item[{Args:}] \leavevmode\begin{itemize}
\item {} 
ut\_date: Refer to \code{Observatory.upd\_date()}

\item {} 
local\_date: Refer to \code{Observatory.upd\_date()}

\item {} 
recalcAll (bool or None) {[}optional{]}: if \code{False} (default): only targets selected for observation are re-processed, if \code{True}: all targets are re-processed, if \code{None}: no re-process

\end{itemize}

\item[{Kwargs:}] \leavevmode
See class constructor

\item[{Raises:}] \leavevmode\begin{itemize}
\item {} 
KeyError: if the twilight keyword is unknown

\item {} 
Exception: if the observatory object has no date

\end{itemize}

\end{description}

\end{fulllineitems}

\index{change\_obs() (astroobs.Observation method)}

\begin{fulllineitems}
\phantomsection\label{astroobs:astroobs.Observation.change_obs}\pysiglinewithargsret{\bfcode{change\_obs}}{\emph{obs}, \emph{long=None}, \emph{lat=None}, \emph{elevation=None}, \emph{timezone=None}, \emph{temp=None}, \emph{pressure=None}, \emph{moonAvoidRadius=None}, \emph{horizon\_obs=None}, \emph{dataFile=None}, \emph{recalcAll=False}, \emph{**kwargs}}{}
Changes the observatory and optionaly re-processes all target for the new observatory and same date
\begin{description}
\item[{Args:}] \leavevmode\begin{itemize}
\item {} 
recalcAll (bool or None) {[}optional{]}: if \code{False} (default): only targets selected for observation are re-processed, if \code{True}: all targets are re-processed, if \code{None}: no re-process

\end{itemize}

\item[{Kwargs:}] \leavevmode
See class constructor

\end{description}

\begin{notice}{note}{Note:}\begin{itemize}
\item {} 
Refer to \code{ObservatoryList.add()} for details on other input parameters

\end{itemize}
\end{notice}

\end{fulllineitems}

\index{plot() (astroobs.Observation method)}

\begin{fulllineitems}
\phantomsection\label{astroobs:astroobs.Observation.plot}\pysiglinewithargsret{\bfcode{plot}}{\emph{y='alt'}, \emph{**kwargs}}{}
Plots the y-parameter vs time diagram for the target at the given observatory and date
\begin{description}
\item[{Kwargs:}] \leavevmode\begin{itemize}
\item {} 
See class constructor

\item {} 
See \code{Observatory.plot()}

\item {} 
moon (bool): if \code{True}, adds the moon to the graph, default is \code{True}

\item {} 
autocolor (bool): if \code{True}, sets curves-colors automatically, default is \code{True}

\end{itemize}

\item[{Raises:}] \leavevmode
N/A

\end{description}

\end{fulllineitems}

\index{polar() (astroobs.Observation method)}

\begin{fulllineitems}
\phantomsection\label{astroobs:astroobs.Observation.polar}\pysiglinewithargsret{\bfcode{polar}}{\emph{**kwargs}}{}
Plots the sky-view diagram for the target at the given observatory and date
\begin{description}
\item[{Kwargs:}] \leavevmode\begin{itemize}
\item {} 
See class constructor

\item {} 
See \code{Observatory.plot()}

\item {} 
moon (bool): if \code{True}, adds the moon to the graph, default is \code{True}

\item {} 
autocolor (bool): if \code{True}, sets curves-colors automatically, default is \code{True}

\end{itemize}

\item[{Raises:}] \leavevmode
N/A

\end{description}

\end{fulllineitems}

\index{rem\_target() (astroobs.Observation method)}

\begin{fulllineitems}
\phantomsection\label{astroobs:astroobs.Observation.rem_target}\pysiglinewithargsret{\bfcode{rem\_target}}{\emph{tgt}, \emph{**kwargs}}{}
Removes a target from the observation list
\begin{description}
\item[{Args:}] \leavevmode\begin{itemize}
\item {} 
tgt (int): the index of the target in the \code{Observation.targets} list

\end{itemize}

\item[{Kwargs:}] \leavevmode
See class constructor

\item[{Raises:}] \leavevmode
N/A

\end{description}

\end{fulllineitems}

\index{targets (astroobs.Observation attribute)}

\begin{fulllineitems}
\phantomsection\label{astroobs:astroobs.Observation.targets}\pysigline{\bfcode{targets}}
Shows the list of targets recorded into the Observation

\end{fulllineitems}

\index{tick() (astroobs.Observation method)}

\begin{fulllineitems}
\phantomsection\label{astroobs:astroobs.Observation.tick}\pysiglinewithargsret{\bfcode{tick}}{\emph{tgt}, \emph{forceTo=None}, \emph{**kwargs}}{}
Changes the ticked property of a target (whether it is selected for observation)
\begin{description}
\item[{Args:}] \leavevmode\begin{itemize}
\item {} 
tgt (int): the index of the target in the \code{Observation.targets} list

\item {} 
forceTo (bool) {[}optional{]}: if \code{True}, selects the target for observation, if \code{False}, unselects it, if \code{None}, the value of the selection is inverted

\end{itemize}

\item[{Kwargs:}] \leavevmode
See class constructor

\item[{Raises:}] \leavevmode
N/A

\end{description}

\begin{notice}{note}{Note:}\begin{itemize}
\item {} 
Automatically reprocesses the target for the given observatory and date if it is selected for observation

\end{itemize}
\end{notice}

\begin{Verbatim}[commandchars=\\\{\}]
\PYG{g+gp}{\PYGZgt{}\PYGZgt{}\PYGZgt{} }\PYG{k+kn}{import} \PYG{n+nn}{astroobs.obs} \PYG{k+kn}{as} \PYG{n+nn}{obs}
\PYG{g+gp}{\PYGZgt{}\PYGZgt{}\PYGZgt{} }\PYG{n}{o} \PYG{o}{=} \PYG{n}{obs}\PYG{o}{.}\PYG{n}{Observation}\PYG{p}{(}\PYG{l+s}{\PYGZsq{}}\PYG{l+s}{ohp}\PYG{l+s}{\PYGZsq{}}\PYG{p}{,} \PYG{n}{local\PYGZus{}date}\PYG{o}{=}\PYG{p}{(}\PYG{l+m+mi}{2015}\PYG{p}{,}\PYG{l+m+mi}{3}\PYG{p}{,}\PYG{l+m+mi}{31}\PYG{p}{,}\PYG{l+m+mi}{23}\PYG{p}{,}\PYG{l+m+mi}{59}\PYG{p}{,}\PYG{l+m+mi}{59}\PYG{p}{)}\PYG{p}{)}
\PYG{g+gp}{\PYGZgt{}\PYGZgt{}\PYGZgt{} }\PYG{n}{o}\PYG{o}{.}\PYG{n}{add\PYGZus{}target}\PYG{p}{(}\PYG{l+s}{\PYGZsq{}}\PYG{l+s}{arcturus}\PYG{l+s}{\PYGZsq{}}\PYG{p}{)}
\PYG{g+gp}{\PYGZgt{}\PYGZgt{}\PYGZgt{} }\PYG{n}{o}\PYG{o}{.}\PYG{n}{targets}
\PYG{g+go}{[Target: \PYGZsq{}arcturus\PYGZsq{}, 14h15m39.7s +19°16\PYGZsq{}43.8\PYGZdq{}, O]}
\PYG{g+gp}{\PYGZgt{}\PYGZgt{}\PYGZgt{} }\PYG{n}{o}\PYG{o}{.}\PYG{n}{tick}\PYG{p}{(}\PYG{l+m+mi}{4}\PYG{p}{)}
\PYG{g+gp}{\PYGZgt{}\PYGZgt{}\PYGZgt{} }\PYG{n}{o}\PYG{o}{.}\PYG{n}{targets}
\PYG{g+go}{[Target: \PYGZsq{}arcturus\PYGZsq{}, 14h15m39.7s +19°16\PYGZsq{}43.8\PYGZdq{}, \PYGZhy{}]}
\end{Verbatim}

\end{fulllineitems}

\index{ticked (astroobs.Observation attribute)}

\begin{fulllineitems}
\phantomsection\label{astroobs:astroobs.Observation.ticked}\pysigline{\bfcode{ticked}}
Shows whether the target was select for observation

\end{fulllineitems}


\end{fulllineitems}



\chapter{setup module}
\label{setup:setup-module}\label{setup::doc}

\chapter{Indices and tables}
\label{index:indices-and-tables}\begin{itemize}
\item {} 
\DUspan{xref,std,std-ref}{genindex}

\item {} 
\DUspan{xref,std,std-ref}{modindex}

\item {} 
\DUspan{xref,std,std-ref}{search}

\end{itemize}



\renewcommand{\indexname}{Index}
\printindex
\end{document}
